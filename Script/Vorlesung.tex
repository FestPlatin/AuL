\documentclass[a4paper,10pt]{scrreprt}
\usepackage[utf8x]{inputenc}
\usepackage[ngerman]{babel}			% deutsche Silbentrennung
\usepackage{listings}				% Codeblock einbinden
\usepackage{color}					% Farbe einbinden
\usepackage{hyperref}				% Hyperlinks hervorheben
\usepackage{tikz}					% Zeichnen
\usepackage{tikz-qtree}				% speziell Zeichen von Bäumen
\usepackage{pgfplots}				% TODO Irgendwas zum zeichnen
\usepackage{makeidx}				% Stichwortverzeichnis einbinden
\usepackage{framed}					% Rahmen für Definitinen
\usepackage{amssymb}   				% Mathematische Symbole und Zeichen
\usepackage{amsmath}   				% Mathematische Symbole und Zeichen
\usepackage{subfig}
\usepackage{adjustbox}
\usepackage{algpseudocode} 
\definecolor{javared}{rgb}{0.6,0,0} % for strings
\definecolor{javagreen}{rgb}{0.25,0.5,0.35} % comments
\definecolor{javapurple}{rgb}{0.5,0,0.35} % keywords
\definecolor{javadocblue}{rgb}{0.25,0.35,0.75} % javadoc
\definecolor{shadecolor}{gray}{0.9}

\setlength{\parindent}{0cm}
\setlength{\parskip}{0.2cm}

\usetikzlibrary{arrows,calc}
\usetikzlibrary{shapes.misc}		% Durchgetichene Nodes
	\usetikzlibrary{positioning}
\tikzset{
	%Define standard arrow tip
	>=stealth',
	%Define style for different line styles
	help lines/.style={dashed, thick},
	axis/.style={<->},
	important line/.style={thick},
	connection/.style={thick, dotted},
}
\usetikzlibrary{calc}
\usetikzlibrary{intersections}
\newcommand\Punkt{\tikz[scale=0.07]\draw[thick](-1,-1)--(1,1)(-1,1)--(1,-1);} 
\lstset{
	language=Java,
	frame=lines, 				% Oberhalb und unterhalb des Listings ist eine Linie
	captionpos=b,                    % sets the caption-position to bottom
	keywordstyle=\color{javapurple}\bfseries,
	stringstyle=\color{javared},
	commentstyle=\color{javagreen},
	morecomment=[s][\color{javadocblue}]{/**}{*/},
	numbers=left,
	numberstyle=\tiny\color{black},
	numbersep=10pt,
	tabsize=4,
	showspaces=false,
	showstringspaces=false}


\tikzset{ none/.style={draw=none} }
\tikzstyle{EndPoint} = [circle, minimum width=4pt,fill, inner sep=0pt]
\tikzstyle{state} = [circle,draw, minimum width=4pt]
\tikzset{cross/.style={cross out, draw=red, minimum size=2*(#1-\pgflinewidth), inner sep=0pt, outer sep=0pt},cross/.default={1pt}}
\tikzstyle{zigzag} = [->,line join=round, decorate, decoration={zigzag, segment length=4, amplitude=1.8, post=lineto, post length=4pt}]
\tikzset{emission/.style={rectangle, draw, inner sep=2pt, minimum size=16pt, node distance=50pt}}

% Stichwortverzeichnis erstellen
\makeindex
% Title Page

\title{Algorithmen und Lernverfahren}
\subtitle{Vorlesung von Prof. Dr. Boris Hollas}
\author{Kay Förster}
\date{\today}

\usepackage{array}
\newcolumntype{M}[1]{>{\centering\arraybackslash}m{#1}}



% Dokument Anfang
\begin{document}
\maketitle

%Inhaltsverzeichnis anzeigen
\tableofcontents

\chapter{O-Notation, Laufzeitanalyse}
Man kann den Zeitaufwand von Algorithmen nicht eindeutig bestimmen.
Viel zu viele Faktoren (Hardware, parallel laufende Programme, Eingabereihenfolge, \dots) spielen eine Rolle, so dass man mit normalen Mitteln niemals eine genaue und allgemeine Aussage über die benötigte Zeit machen kann.
Es werden nun nicht mehr die benötigten Zeiten, sondern die benötigten ``greifbaren'' Schritte bei einer bestimmten Eingabelänge n beschrieben.
Somit können Programme in Klassen (konstant, logarithmisch, lineas, polynomial, exponentiell, u.a.) eingeteilt werden.

\section{O-Notation}
Wir verwenden die \(\mathcal{O}\)-Notation um die Laufzeit und den Platzbedarf von Algorithmen anzugeben.
Dazu betrachten wir die maximale Anzahl Schritte, die ein Algorithmus ausführt.

\subsubsection{Beispiel}
Es soll die Laufzeit der lineare Suche berechnet werden.
\begin{lstlisting}[language=java, caption={Pseudocode zur Berechnung der Laufzeit}]
for (k := 1 to n) {
	if (a[k] == gesuchter Wert)
		return true;
}
return false;
\end{lstlisting}
Lösung:
\begin{eqnarray*}
	LZ	&\leq& n \cdot c_{1} + c_{2} \\
		&\leq& n \cdot c + c  \textrm{ wobei } c= \max\{c_{1},c_{2}\} \\
		&\leq& n \cdot c + n \cdot c	\\
		&=& 2c \cdot n \in \mathcal{O}(n)
\end{eqnarray*}
Die Laufzeit der linearen Suche ist \(\leq c \cdot n\), wobei \(c\) eine Konstante ist, die von der Implentierung und dem Computer abhängt.

\begin{shaded}
	\noindent
	\textbf{Def.:} Für eine Funktion \(f \geq 0\) ist \(\mathcal{O}(f)\) die Menge aller Funktionen \(g\) mit 
		\[0 \leq g(n) \leq c \cdot f(n)\] 
		für eine Konstante \(c > 0\) für alle hinreichend großen \(n\).
		
		\(\mathcal{O}(f) = \{ g\:|\: 0 \leq g(n) \leq c \cdot f(n)\) für ein \(c > 0\) und allen großen \(n \in \mathcal{N}\}\)
\end{shaded}
Die \(\mathcal{O}\)-Notation stellt somit die maximale Laufzeit (Worst Case Laufzeit) eines Algorithmuses dar.
Die Funktion \(g(n)\) ist die konkrete Laufzeit einer gegebenen Implentierung.
\begin{figure}[htbp]
	\begin{center}
		\begin{tikzpicture}[
				%We set the scale and define some styles
				scale=1.5,
				axis/.style={very thick, ->, >=stealth'},
				important line/.style={thick},
				dashed line/.style={dashed, thick},
				every node/.style={color=black,}
			]
			
			% Important coordinates are defined
			\coordinate (x11) at (0,0);
			\coordinate (x12) at (2,1.5);
			\coordinate (x21) at (0,0.2);
			\coordinate (x22) at (2.5,1.3);
		
			\begin{scope}
			\shade[top color=white, bottom color=red]
			(x12) parabola bend (0,0) (2,0);
			\end{scope}

			% axis
			\draw[axis] (0,0)  -- (3,0) node(xline)[right] {n};
			\draw[axis] (0,0) -- (0,2) node(yline)[above] {};
			
			% J curve is drawn
			\draw[important line]
			(x11) parabola (x12) node[above] {$c \cdot f$}
			(x21) parabola (x22) node[right] {g};

			\draw[dashed line] (1.8,.3) -- (2.5,.5) node[right] {$\mathcal{O}(f)$};
			\draw[thick] (1,.38) circle (1.5pt) node[above] {$n_{0}$};
		\end{tikzpicture}
	\end{center}
	\label{img:ONotation}
	\caption{Grafische Darstellung der \(\mathcal{O}\)-Funktion}
\end{figure}

Es gelten folgende Rechenregeln:
\begin{itemize}
	\item \(\mathcal{O}(f) + \mathcal{O}(g) = \mathcal{O}(\max\{f,g\})\)
	\item \(\mathcal{O}(f) + \mathcal{O}(g) = \mathcal{O}(f + g)\)
	\item \(\mathcal{O}(f) \cdot \mathcal{O}(g) = \mathcal{O}(f \cdot g)\)
	\item \(\mathcal{O}(c \cdot f) = \mathcal{O}(f)\) (für alle \(c \geq 0)\)
	\item \(f \leq g \Rightarrow \mathcal{O}(f) \subseteq \mathcal{O}(g)\)
	\item \(\mathcal{O}(c) = \mathcal{O}(1)\)
	\item \(c = \mathcal{O}(1)\)
\end{itemize}


\subsubsection{Übung}
\begin{eqnarray*}
	0	&\leq& 17n^3 + 5n^2 + 2n + 8 \\
		&\leq& 17n^3 + 5n^3 + 2n^3 + 8n^3 \\
		&\leq& 32n^3 \in \mathcal{O}(n^3)
\end{eqnarray*}

\subsubsection{Übung}
Berechnen Sie die Laufzeit des folgenden Codes:
\begin{lstlisting}[language=java, caption={Pseudocode zur Berechnung der Laufzeit}]
	for (k := 1 to n-1)
		for (l := k+1 to n)
			if (a[k] == a[l])
				return Duplikat vorhanden;
	return Kein Duplikat vorhanden;
\end{lstlisting}
\begin{enumerate}
	\item Möglichkeit zur Abschätzung der Laufzeit
		\begin{eqnarray*}
		LZ 	&\leq& n \cdot n \cdot c + c'	\\
			&\leq& n^2 \cdot c + n^2 \cdot c'	\\
			&\leq& (c+c') n^2				\\
			&\leq& \mathcal{O}(n^{2})
		\end{eqnarray*}
	\item Möglichkeit zur genaueren Abschätzung der Laufzeit
	\begin{eqnarray*}
		LZ	&\leq& \binom{n}{2} \cdot c + c' \\
			&\leq& \binom{n}{2} \cdot c + \binom{n}{2} \\
			&=& \binom{n}{2}(c+1)	\\
			&=& \frac{n(n-1)}{2}(c+1)	\\
	\curvearrowright LZ &\leq& n^{2} \cdot \frac{c+1}{2} \in \mathcal{O}(n^{2})
	\end{eqnarray*}
\end{enumerate}

\newpage
\section{Omega-Notation}
\label{sec:OmegaNotation}
Die Omega-Notation beschreibt die untere Schranke, d.h. wie lange ein Algorithmus bzw. ein Programm mindestens läuft (Best Case Laufzeit).
\begin{shaded}
	\noindent
	\textbf{Def.:} \(\Omega(f) = \{ g \:|\) Es gibt ein \(c > 0\), sodass \(0 \leq c \cdot f(n) \leq g(n)\) für alle großen \(n\) gilt.\(\}\)
\end{shaded}

\begin{figure}[htbp]
	\begin{center}
		\begin{tikzpicture}[
				%We set the scale and define some styles
				scale=1.5,
				axis/.style={very thick, ->, >=stealth'},
				important line/.style={thick},
				dashed line/.style={dashed, thick},
				every node/.style={color=black,}
			]
			
			% Important coordinates are defined
			\coordinate (x11) at (0,0);
			\coordinate (x12) at (2,1.5);
			
			\begin{scope}
			\shade[top color=white, bottom color=red]
			(x12) parabola bend (0,0) (0,1.5);
			\end{scope}
			
			% axis
			\draw[axis] (0,0)  -- (3,0) node(xline)[right] {n};
			\draw[axis] (0,0) -- (0,2) node(yline)[above] {};
			
			% J curve is drawn
			\draw[important line]
			(x11) parabola (x12) node[above] {$c \cdot f$};

			\draw[dashed line] (1,1.3) -- (1,1.7) node[above] {$\Omega(f)$};
		\end{tikzpicture}
	\end{center}
	\label{img:OmegaNotation}
	\caption{Grafische Darstellung der Omega-Funktion}
\end{figure}

\section{Theta-Notation}
\label{sec:ThetaNotation}
Die Theta-Notation dient dazu, gleichzeitg eine obere und eine untere Schranke zu definieren.
\begin{shaded}
	\noindent
	\textbf{Def.:} \( \Theta(f) = \mathcal{O} \cap \Omega(f)\)
\end{shaded}
\begin{figure}[htbp]
	\begin{center}
		\begin{tikzpicture}[
				%We set the scale and define some styles
				scale=1.5,
				axis/.style={very thick, ->, >=stealth'},
				important line/.style={thick},
				dashed line/.style={dashed, thick},
				every node/.style={color=black,}
			]
			
			% Important coordinates are defined
			\coordinate (x11) at (0,0);
			\coordinate (x12) at (1.5,1.5);
			\coordinate (x22) at (2.5,1.5);
			
			\begin{scope}
				\shade[top color=white, bottom color=red]
					(x22) parabola bend (0,0) (x12);
			\end{scope}
			
			% axis
			\draw[axis] (0,0)  -- (3,0) node(xline)[right] {n};
			\draw[axis] (0,0) -- (0,2) node(yline)[above] {};
			
			% J curve is drawn
			\draw[important line]
			(x11) parabola (x12) node[above] {$c_{1} \cdot f$};
			\draw[important line]
			(x11) parabola (x22) node[above] {$c_{2} \cdot f$};

			\draw[dashed line] (1.5,.8) -- (2.3,.8) node[right] {$\Theta(f))$};
		\end{tikzpicture}
	\end{center}
	\label{img:Theta}
	\caption{Grafische Darstellung der Theta-Funktion}
\end{figure}

\chapter{Suchen und Sortieren}
\section{Lineare Suche}
\label{sec:lineareSuche}
Die Lineare Suche ist der einfachste Suchalgorithmus überhaupt.
Bei ihr wird solange ein Element nach dem anderen durchlaufen, bis ein Element mit dem gesuchten Schlüssel angetroffen wird.
Die lineare Suche hat eine Laufzeit von $\mathcal{O}(n)$ (n ist die Anzahl der Elemente der Liste) und kann sowohl auf sortierte als auch unsortierte Listen angewendet werden. 
\begin{lstlisting}[language=java, caption={Beispielimplementierung in Java}]
public static int lineareSuche(int gesucht, int[] daten) {
    for (int i = 0; i < daten.length; i++)
        if (daten[i] == gesucht)
            return i;
    return -1;
}
\end{lstlisting}

\section{Binäre Suche}
\label{sec:binaereSuche}
Die binäre Suche ist ein Algorithmus, der in einem Array sehr effizient ein gesuchtes Element findet bzw. eine zuverlässige Aussage über das Fehlen dieses Elementes liefert.
Voraussetzung ist, dass die Elemente in dem Array entsprechend sortiert sind.

Dazu wird immer das mittlere Element eines Felds überprüft.
Ist das Element gleich dem gesuchten Element ist die Suche beendet.
Ansonsten wird geprüft ob das Element kleiner als das gesuchte Element ist, dann muss sich das Element in der vorderen Hälfte befinden, ansonsten in der hinteren.
Dadurch wird der Suchbereich Schritt für Schritt halbiert bis das gesuchte Element gefunden ist oder nur noch eine Element vorhanden ist.

Für \(n=2^{k}\) lässt sich das Verhalten bei erfolgloser Suche als vollständiger Binärbaum darstellen.
Jeder Knoten entspricht ein Vergleich mit einem mittleren Feldelement.
Jedes Blatt entspricht einem Vergleich in einem Array der Länge 1, daher besitzt der Baum \(n\) Blätter.
\begin{figure}[htbp]
	\begin{center}
	\tikzstyle{end} = [circle, minimum width=4pt,fill, inner sep=0pt]
		\begin{tikzpicture}[sibling distance=5mm]
		\Tree [
		  . \node[end]{};
		      [. \node[end]{}; 
			    [.\node[end]{};
				  [. \node[end](a){}; ]
				  [. \node[end]{}; ]
			    ]
			    [. \node[end]{};
				  [. \node[end]{}; ]
				  [. \node[end]{}; ]
			    ]
		      ]
		      [. \node[end]{};
			    [.\node[end]{};
				  [. \node[end]{}; ]
				  [. \node[end]{}; ]
			    ]
			    [. \node[end]{};
				  [. \node[end]{}; ]
				  [. \node[end](z){}; ]
			    ]
		      ]
		]
		\draw[thick,decorate,decoration={brace,amplitude=12pt,mirror}] ($(a)+(-.2,0)$) -- ($(z)+(.2,0)$) node[midway, yshift=-20pt]{n};
		\end{tikzpicture}
	\end{center}
	\label{img:BinaerBaum}
	\caption{Einfacher Binärbaum}
\end{figure}
In einem vollständigen Binärbaum mit genau \(n=2^{k}\) Blättern besitzt jeder Pfad von der Wurzel zu einem Blatt die Länge \(k=\log_{2}n\).
Falls die Suche erfolgreich ist, ist die Laufzeit entsprechend kürzer.
Die Laufzeit der binären Suche liegt daher in \(\mathcal{O}(\log_{2}n)\).

\subsubsection{Induktionsbeweis}
\label{Beweis:BinaerBaum}
Behauptung: Ein vollständiger Binärbaum der Tiefe \(d\) besitzt genau \(2^{d}\) Blätter.
\begin{description}
 \item[\(d=0\)] Ein vollständiger Binärbaum der Tiefe 0 besitzt \(1=2^{0}\) Blätter.
	\item[\(d\rightarrow d+1\)] Wenn wir in einem vollständigen Binärbaum der Tiefe \(d+1\) die Wurzel entfernen, erhalten wir zwei vollständige Binärbäume der Tiefe \(d\).
			Diese besitzen nach Induktionsvorraussetzung jeweils genau \(2^{d}\) Blätter.
			Daraus folgt, dass der Binärbaum der Tiefe \(d+1\) genau \(2 \cdot 2^{d} = 2^{d+1}\) Blätter besitzt.
\end{description}

\subsubsection{Rekursionsgleichung}
Eine alternative Möglichkeit die Laufzeit des Algorithmuses zu bestimmen ist es die Rekursionsgleichung zu lösen.
Seien \(n=2^{k}\) und \(V(n)\) die Anzahl, die die binäre Suche mit dem gesuchten Wert ausführt.
So gilt bei erfolgloser Suche:

\begin{eqnarray*}
	V(1) &=& 1 \\
	V(n) &=& 1+ V(\frac{n}{2}) \\
	&=& 1+1+V(\frac{n}{4}) \\
	&=& 1+1+1+V(\frac{n}{8}) = 3 +V(\frac{n}{2^{3}}) \\
	&=& k+V(\frac{n}{2^{k}}) = \log(n) + V(\frac{n}{n})\\
	&=& \log(n) + V(1) \in \mathcal{O}(\log n)
\end{eqnarray*}

\subsubsection{Alternative Herleitung}
Eine weitere Alternative Herleitung geht über die Herleitung einer Schleife:
\[ \textrm{Anzahl Schleifendurchläufe} \cdot \textrm{Aufwand pro Schleife} \]
\[ \mathcal{O}(k) \cdot \mathcal{O}(1) = \mathcal{O}(k) = \mathcal{O}(\log n) \]

\section{Binäre Suchbäume}
Um auch in dynamischen Datenstrukturen zu suchen, lassen sich Suchbäume verwenden.
Ein Suchbaum ist ein Binärbaum in dem gilt: Jeder in einem Knoten gespeicherte Wert ist größer als alle Knoten im linken Teilbaum und kleiner als alle Knoten im rechten Teilbaum.
\begin{figure}[htbp]
	\begin{center}
		\begin{tikzpicture}
		\Tree [
			.5
			[.3
				[.2 ]
				[.4 ]
			]
			[.8
				[.6 ]
				[.9 ]
			]
		]
		\end{tikzpicture}
	\end{center}
	\label{img:SuchBaum}
	\caption{Beispiel für einen Suchbaum}
\end{figure}
Folgende Funktionen sind nötig:
\begin{description}
	\item[Suchen nach einem Wert] Dabei wird der Suchbaum, beginnend an der Wurzel, rekursiv durchgesucht.
		Die Laufzeit liegt bei \(\mathcal{O}(n)\) für einen linear entarteten Baum und \(\mathcal{O}(\log n)\) für einen vollständigen Baum.
	\item[Wert hinzufügen] Dazu wird der Baum wie oben durchsucht und ein Blatt mit dem neuen Wert hinzugefügt falls der Wert noch nicht vorhanden ist.
		Die Laufzeit ist gleich der des durchsuchens eines Baumes.
	\item[Suchbaum aufbauen] Dazu kann mehrfach die obrige Funktion aufgerufen werden.
		Jedoch kann dabei ein unbalancierter Baum entstehen.
		Es gibt einen Algorithmus der einen optimalen Suchbaum aufbaut.
	\item[Wert entfernen] Es werden im allgemeinen zwei Fälle unterschieden:
		\begin{description}
			\item[einfacher Fall] Der zu entfernende Knoten hat keine oder genau einen Nachfolger
			\item[schwieriger Fall] Knoten besitzt zwei Nachfolger.
				Eine einfache Lösung ist es, den zu entfernenden Knoten mit dem kleinsten Knoten im rechten Unterbaum zu ersetzen.
				Dazu wird der Knoten mit minimalen Wert im rechten Unterbaum gesucht und durch einen rekursiven Aufruf entfernt und als eine Wurzel angehängt.
		\end{description}
\end{description}

\begin{figure}[htbp]
	\begin{center}
		\subfloat[Ausgangssuchbaum]{
			\begin{tikzpicture}
			\tikzset{ n/.style={draw=none} }
			\Tree [
				.5
				[.2
					\edge[n];[.{} ]
					[.4 
						[.3 ]
						\edge[n];[.{} ]
					]
				]
				[.8
					[.6
						\edge[n];[.{} ]
						[.7 ]
					]
					[.10
						[.9 ]
						[.11 ]
					]
				]
			]
			\end{tikzpicture}
		}
		\hspace{1cm}
		\subfloat[Resultat]{
			\begin{tikzpicture}
			\tikzset{ n/.style={draw=none} }
			\Tree [
				.6
				[.2
					\edge[n];[.{} ]
					[.4 
						[.3 ]
						\edge[n];[.{} ]
					]
				]
				[.8
					[.7 ]
					[.10
						[.9 ]
						[.11 ]
					]
				]
			]
			\end{tikzpicture}
		}
	\end{center}
	\label{img:SuchBaumAdd}
	\caption{Beispiel für das entfenen eines Wertes aus einem Suchbaum}
\end{figure}

Mit einer Grammatik lässt sich ein Binärbaum darstellen:
\[\textrm{BTree} \rightarrow \textrm{empty} \:|\: \textrm{node Btree Btree}\]
Zum Beispiel entspricht die Grammatik ``node(node empty empty) empty'' dem Baum
\begin{figure}[htbp]
	\begin{center}
	\tikzstyle{end} = [circle, minimum width=4pt,fill, inner sep=0pt]
		\begin{tikzpicture}[sibling distance=5mm]
			\Tree [. \node[end]{};
				[. \node[end]{};
					[. \node[end]{};
					]
					[. \node[end]{};
					]
				]
				[. \node[end]{};
				]
				];
		\end{tikzpicture}
	\end{center}
\end{figure}


\section{Hashing}
Gegeben sei eine Menge \(U\) von potentiellen Schlüsseln und eine Menge \(S \subseteq U\) von zu verwaltenden Schlüsseln.
Hashing ist geeignet, wenn \(|S|\) deutlich kleiner \(|U|\) ist und sich \(|S|\) nicht stark ändert.
Beispiel:
\begin{itemize}
	\item U \ldots alle möglichen ISBN-Nummern
	\item S \ldots tatsächlich in einer Buchhandlung vorkomenden ISBN-Nummern
\end{itemize}

Wir verwenden eine Hashfunktion \(h: U \rightarrow T\), die eine Hashtabelle \(T\) abbildet.
Ein Beispiel für eine Hashfunktion könnte \(h(s) = s \mod m\) sein, wenn \(m \leq |T|\) ist.
\(m\) sollte dabei eine Primzahl sein.
\begin{figure}[htbp]
	\begin{center}
		\begin{tikzpicture}[every node/.style={circle,inner sep=0pt}]
			\draw (0,0) -- (3,0) -- (3,3) -- (0,3) -- (0,0);
			\draw (1.5,1.5) circle [radius=.3] node {S};
			\draw (.2,.2) node {U};
			\draw (5,0) -- (6,0) -- (6,3) -- (5,3) -- (5,0);
			\draw (5.5,.2) node {\(n-1\)};
			\draw (5.5,2.6) node {0};
			\draw (5.5,2) node {1};
			\draw (5.5,1.2) node {\(\vdots\)};
			\draw (5,2.4) -- (6,2.4);
			\draw (5,1.8) -- (6,1.8);
			\draw (5,.6) -- (6,.6);
			\draw [->] (1.8,1.5) -- (5,2.6);
			\draw [->] (1.8,1.5) -- (5,2);
			\draw [->] (1.8,1.5) -- (5,.2);
		\end{tikzpicture}
	\end{center}
	\label{img:Hashfunktion}
	\caption{Hashfunktion die auf ein Array abbildet}
\end{figure}

Dabei können allerdings Kollisionen (mehre Schlüssel besitzen den gleichen Hashwert) auftretten.
Dies soll durch das folgende Beispiel verdeutlicht werden:
In einer Hashtabelle mit \(m=100\) Einträgen werden \(k\) zufällige Schlüssel eingetragen.
Wir nehmen an, dass die Hashfunktion gut streut und die Werte der Hashfunktion gleichmäßig verteilt sind.
Ab wann ist die Wahrscheinlickeit für eine Kollision \(\geq 0,5\)?
\begin{eqnarray*}
	P(Kollesion) &=& 1-(\textrm{kleine Kollesion}) \\
				&=& 1-1\cdot \frac{99}{100} \cdot \frac{98}{100} \cdot \ldots \cdot \frac{100-k+1}{100} > \frac{1}{2} \\
				&\Leftrightarrow& k \geq 13
\end{eqnarray*}
Allgemein ist bei etwa \(\sqrt{2m}\) vielen Einträgen mit einer Kollesion zu rechnen.

Die einfachste Art der Kollisionsbehandlung ist das Hashing mit Verkettung.
Dabei ist jedes Element der Hashtabelle eine Liste.
\begin{figure}[htbp]
	\begin{center}
		\begin{tikzpicture}[every node/.style={circle,inner sep=0pt}]
			\draw (0,0) -- (.5,0) -- (.5,3) -- (0,3) -- (0,0);
			\draw (0,.5) -- (.5,.5);
			\draw (0,1) -- (.5,1);
			\draw (0,1.5) -- (.5,1.5);
			\draw (0,2) -- (.5,2);
			\draw (0,2.5) -- (.5,2.5);
			\draw [->] (.25,.25) -- (1,.25);
			\draw (1,0) -- (1.5,0) -- (1.5,.5) -- (1,.5) -- (1,0);
			\draw [->] (1.25,.25) -- (2,.25);
			\draw (2,0) -- (2.5,0) -- (2.5,.5) -- (2,.5) -- (2,0);
			\draw [->] (2.25,.25) -- (3,.25);
			\draw (3,0) -- (3.5,0) -- (3.5,.5) -- (3,.5) -- (3,0);
			\draw [->] (.25,1.25) -- (1,1.25);
			\draw (1,1) -- (1.5,1) -- (1.5,1.5) -- (1,1.5) -- (1,1);
			\draw [->] (.25,2.25) -- (1,2.25);
			\draw (1,2) -- (1.5,2) -- (1.5,2.5) -- (1,2.5) -- (1,2);
			\draw [->] (.25,2.75) -- (1,2.75);
			\draw (1,2.5) -- (1.5,2.5) -- (1.5,3) -- (1,3) -- (1,2.5);
			\draw [->] (1.25,2.75) -- (2,2.75);
			\draw (2,2.5) -- (2.5,2.5) -- (2.5,3) -- (2,3) -- (2,2.5);
		\end{tikzpicture}
	\end{center}
	\label{img:HashVerkettung}
	\caption{Hashing mit Verkettung; alle Daten, deren Schlüssel auf denselben Hashwert führen, werden in die entsprechende Liste eingetragen}
\end{figure}

Der Belegungsfaktor einer Hastabelle mit  \(n\) Elementen ist \(\beta = \frac{n}{m}\).
Man kann zeigen, dass die mittlere Länge der Überlaufliste \(\beta\) ist.
Daraus ergibt sich die mittlere Anzahl Suchschritte \(1+\beta\).

\subsubsection{Dynamische Größenänderung}
Um die Suchzeit und den Platzbedarf gering zu halten, muss der Belegungsquotient begrenzt sein.
Eine einfache Möglichkeit ist es das beim Überschreiten eines Schwellwertes für \(\beta\) (typisch \(\beta = \frac{3}{4}\)) alle Einträge in eine größere Hashtabelle kopiert werden.
Entsprechend beim Unterschreiten eine Schwellwertes in eine kleinere Hashtabelle.
Mit einer amortisierten Analyse lässt sich zeigen, dass sich der amortisierte Aufwand (d.h. wir verteilen den Aufwand zum Kopieren auf alle vorherigen Hashoperationen) für die Hashtabelle dann in \(\mathcal{O}(1)\) liegt.


\section{Sortieren}

\begin{shaded}
  \noindent
  \textbf{Satz.:} In einem Binärbaum mit mind. \(2^{k}\) Blättern gibt es einen Pfad der Länge \(k\).
\end{shaded}
Der Satz lässt sich mit Hilfe des Beweises im Kapitel \ref{Beweis:BinaerBaum} indirekt beweisen.
\[a \rightarrow b \equiv \neg b \rightarrow \neg a\]
Dies würde bedeuten wenn alle Pfade kürzer als \(k\) sind, besitzt der Baum \(< 2^{k}\) Blätter, was zu einem Widerspruch führt.

Ein Sortierverfahren, dass ausschließlich paarweise Vergleiche verwendet, lässt sich als Binärbaum darstellen.
In jedem Knoten wird ein Vergleich \(a \leq b\) ausgeführt.
Jedes Blatt entspricht einer sortierten Folge,
Jede sortierte Folge entspricht einer Permutation der Ausgangsfolge.
Der Binärbaum besitzt daher \(n!\) Blätter und deshalb einen Pfad der Länge \(\log_{2}n!\).
Ein Sortierverfahren, dass paarweise Vergleiche ausführt besitzt daher eine Worst-Case Laufzeit in \(\Omega(\log_{2}n!) = \Omega(n\log_{2}n) - \Theta(n))\).
\begin{figure}[htbp]
	\begin{center}
	\tikzstyle{end} = [circle, minimum width=4pt,fill, inner sep=0pt]
		\begin{tikzpicture}[sibling distance=5mm]
			\Tree [. 1:2
				\edge node[auto=right]{\(\leq\)};
				[.2:3
					\edge node[auto=right]{\(\leq\)}; \(\{1,2,3\}\)
					\edge node[auto=left]{\(>\)};
					[.1:3
						\edge node[auto=right]{\(\leq\)}; \(\{1,3,2\}\)
						\edge node[auto=left]{\(>\)}; \(\{3,1,2\}\)
					]
				]
				\edge node[auto=left]{\(>\)}; 
				[.1:3
					\edge node[auto=right]{\(\leq\)}; \(\{2,1,3\}\)
					\edge node[auto=left]{\(>\)};
					[.2:3
						\edge node[auto=right]{\(\leq\)}; \(\{2,3,1\}\)
						\edge node[auto=left]{\(>\)}; \(\{3,2,1\}\)
					]
				]
				];
\end{tikzpicture}
	\end{center}
	\label{img:Suchnaum}
	\caption{Entscheidungsbaum für 3 Elemente (\(\{1,2,3\}\))}
\end{figure}

Naive Verfahren wie Bubbelsort\footnote{\url{https://www.youtube.com/watch?v=ARZLBeagiJ4}} haben ein Laufzeit von \(\mathcal{O}(n^2)\).
Ein besseres Verfahren ist Quicksort\footnote{\url{https://www.youtube.com/watch?v=UoJJ78K-uc0}}.
Die Laufzeit ist schwierig zu berechnen da die Teillisten unterschiedliche Längen besitzen.
Ein ähnliches Verfahren ist Mergesort, welches nachfolgend behandelt wird.


\subsection{Mergesort}
Mergesort\footnote{\url{https://www.youtube.com/watch?v=yKgzwtqWvFU}} betrachtet die zu sortierenden Daten als Liste und zerlegt sie in kleinere Listen, die jede für sich sortiert werden.
Die sortierten kleinen Listen werden dann zu größeren Listen zusammengefügt, bis wieder eine sortierte Gesamtliste erreicht ist.
Das Verfahren arbeitet bei Arrays in der Regel nicht in-place.

\begin{figure}[htbp]
	\begin{center}
	\tikzstyle{end} = [circle, minimum width=4pt,fill, inner sep=0pt]
		\begin{tikzpicture}[sibling distance=5mm]
			\Tree [. \node[end]{};
				\edge node[auto=right]{\(\frac{n}{2}\)};
				[. \node[end]{};
					\edge node[auto=right]{\(\frac{n}{4}\)};
					[. \node[end]{};
						\node[end]{};
						\node[end]{};
					]
					\edge node[auto=left]{\(\frac{n}{4}\)};
					[. \node[end]{};
						\node[end]{};
						\node[end]{};
					]
				]
				\edge node[auto=left]{\(\frac{n}{2}\)};
				[. \node[end]{};
					\edge node[auto=right]{\(\frac{n}{4}\)};
					[. \node[end]{};
						\node[end]{};
						\node[end]{};
					]
					\edge node[auto=left]{\(\frac{n}{4}\)};
					[. \node[end]{};
						\node[end]{};
						\node[end]{};
					]
				]
				];
		\end{tikzpicture}
	\end{center}
	\caption{Darstellung als Binärbaum}
	\label{fig:Mergesort}
\end{figure}
Das Verhalten lässt sich wie in Abbildung \ref{fig:Mergesort} als Baum darstellen.
Vereinfacht nehmen wir \(n=2^{k}\) an.

Beim Zusammenfügen fällt der Aufwand \(\mathcal{O}((\textrm{linke Liste}) + (\textrm{rechte Liste}))\) an.
Auf jeder Ebene ist das \(\mathcal{O}(n)\).
Zum halbieren der Liste fällt jeweils der Aufwand \(\mathcal{O}(n)\) an.
Pro Ebene insgesammt also \(\mathcal{O}(n)\).
Der Baum besitzt \(n\) Blätter, die Liste der der Länge 1 entsprechen (Mehr als n Listen der Länge 1 können nicht erzeugt werden \(\curvearrowright\) n Blätter).
Der Baum hat daher die Tiefe \(\log_{2}n\) (d.h. k).
Die Laufzeit liegt daher in in \(\mathcal{O}(n \cdot  \log n)\).
Auch die mittlere Laufzeit von Quicksort liegt in \(\mathcal{O}(n \log n)\).

\subsubsection{Übung}
Bestimmen Sie die Laufzeit durch eine Rekursionsgleichung.
\(V(n)\) ist dabei die Anzahl der Vergleiche.
\begin{eqnarray*}
	V(1) &=& 0 \\
	V(n) &=& \underbrace{n}_{\textrm{Zusammenfügen}} + \underbrace{V(\frac{n}{2}) + V(\frac{n}{2})}_{\textrm{Sortieren}} = n + 2V(\frac{n}{2}) \\
	V(n+1) &=& 2n + 4V(\frac{n}{4}) \\
		&\vdots&	\\
		&=& k \cdot n + 2^{k} \cdot V(\frac{n}{2^{k}})	\\
		&=&n \cdot \log n
\end{eqnarray*}

\subsubsection{Worst Case Laufzeit von Quicksort}
Im Worst Case wird das Pivotelement stehts so gewählt, dass es das größte oder das kleinste Element der Liste ist.
Dies ist etwa der Fall, wenn als Pivotelement stets das Element am Ende der Liste gewählt wird und die zu sortierende Liste bereits sortiert vorliegt.
\begin{eqnarray*}
	V(n) &=& n-1 + V(n-1)	\\
		 &=& n-1 + n-2 + V(n-2)	\\
		 &=& \sum \limits_{k=1}^{n-1} k + V(0) = \frac{n(n-1)}{2} \in \mathcal{O}(n^{2})
\end{eqnarray*}

\subsection{Heap Sort}
Ein Heap ist ein Binärbaum, in dem jeder Knoten einen kleineren (Min-Heap) bzw. einen größeren (Max-Heap) Wert besitzt als seine Nachfolger (Abbildung \ref{fig:MinHeap}).
Ein Linksbaum ist ein Heap, der effiziente Heapoperationen ermöglicht.
\begin{figure}[htbp]
	\begin{center}
		\begin{tikzpicture}
		\Tree [
			.1
			[.2
				[.3 ]
				[.8 ]
			]
			[.6
				[.9 ]
				[.7 ]
			]
		]
		\end{tikzpicture}
	\end{center}
	\caption{Beispiel für einen Min-Heap}
	\label{fig:MinHeap}
\end{figure}

Wir stellen Binär\-bäume als erweiterte Binär\-bäume dar, so dass jeder innere Knoten genau 2 Nachfolger hat  (Abbildung \ref{fig:BinaerBaumExtendet}).
Für einen Knoten \(x\) ist \(s(x)\) die Länge des kürzesten Pfades von x zu einem Blatt.
Ein Binärbaum ist ein Linksbaum wenn für jeden inneren Knoten \(x\) gilt:
\[s(\textrm{linker Nachfolger}(x)) \geq s(\textrm{rechter Nachfolger}(x)) \]
\begin{figure}[htbp]
	\begin{center}
 		\subfloat[Binärbaum]{
			\trimbox{-.5cm 0cm -.5cm 0cm}{
				\begin{tikzpicture}[sibling distance=5mm]
					\tikzset{ n/.style={draw=none} }
					\tikzstyle{end} = [circle,draw, minimum width=4pt]
					\Tree [
						.\node[end]{};
						[.\node[end]{}; ]
						\edge[n];[.{} ]
					]
				\end{tikzpicture}
			}
		}
		\hspace{1cm}
		\subfloat[erweiterte Binärbaum]{
			\trimbox{-1cm 0cm -1cm 0cm}{
				\begin{tikzpicture}[sibling distance=5mm]
						\tikzstyle{end} = [circle, minimum width=4pt,fill, inner sep=0pt]
						\Tree [
							.\node[state]{};
								[.\node[state]{};
									[.\node[end]{}; ]
									[.\node[end]{}; ]
								]
							.\node[state]{};
								[.\node[end]{}; ]
						]
				\end{tikzpicture}
			}
		}
	\end{center}
	\caption{Erweiterung eines Binärbäumes}
	\label{fig:BinaerBaumExtendet}
\end{figure}

\newpage
\subsubsection{Eigenschaften}
\begin{enumerate}
	\item \label{itm:first} \(s(root)\) ist die Länge des Pfades ganz rechts.
	\item \label{itm:second} Für die Anzahl \(n\) der inneren Knoten  gilt:
		\[ n\geq \sum \limits_{k=0}^{s(root)-1} 2^{k} = 2^{s(root)}-1\]
	\item Aus \ref{itm:first}, \ref{itm:second} folgt: Die Länge des Pfades ganz rechts liegt in \(\mathcal{O}(\log n)\).
\end{enumerate}

\subsubsection{Operationen}
\begin{itemize}
	\item put
	\item removeMin
	\item init
	\item merge
\end{itemize}

\subsubsection{Merge-Operation für einen Linksbaum}
Wir suchen den Baum mit dem kleineren Wurzelwert und betrachten dessen rechten Teilbaum.
Merge wird wird rekursiv aufgerufen für diesen rechten Teilbaum und den anderen Baum.

\begin{tabular}{M{.45\textwidth}M{.45\textwidth}}
% 0. Zeile
	\multicolumn{2}{l}{\rule{0pt}{5ex}Gegebene Bäume:}
	\\
% 1. Zeile
	\begin{tikzpicture}[sibling distance=5mm]
		\Tree [
			.\node[state]{4};
			[.\node[state]{6};
				[.\node[state]{8}; ]
				[.\node[state]{6}; ]
			]
			[.\node[state]{8};	]
		]
	\end{tikzpicture}
	&
	\begin{tikzpicture}[sibling distance=5mm]
		\Tree [
			.\node[state]{3};
			[.\node[state]{5};
				[.\node[state]{7}; ]
				\edge[none];[.{} ]
			]
			[.\node[state]{6}; 	]
		]
	\end{tikzpicture}
	\\
		\multicolumn{2}{l}{\rule{0pt}{5ex}Begin der Rekursion:}
	\\
% 2. Zeile
	\begin{tikzpicture}[sibling distance=5mm]
		\Tree [
			.\node[state]{4};
			[.\node[state]{6};
				[.\node[state]{8}; ]
				[.\node[state]{6}; ]
			]
			[.\node[state]{8}; ]
		]
		\end{tikzpicture}
	&
	\begin{tikzpicture}[sibling distance=5mm]
		\Tree [.\node[state]{6}; ]
	\end{tikzpicture}
	\\
	\multicolumn{2}{l}{\rule{0pt}{5ex}Nächster Schritt:}
	\\
% 3. Zeile
	\begin{tikzpicture}[sibling distance=5mm]
		\Tree [.\node[state]{8}; ]
	\end{tikzpicture}
	&
	\begin{tikzpicture}[sibling distance=5mm]
		\Tree [.\node[state]{6}; ]
	\end{tikzpicture}
\end{tabular}

Hier endet die Rekursion, da der rechte Teilbaum des Baumes mit der kleineren Wurzel leer ist.
Unter dem Baum mit der kleinen Wurzel wird rechts der andere Baum gehängt.
Wenn dabei die Linksbaumeigenschaft verletzt wird, werden die Teilbäume vertauscht.

\begin{tabular}{M{.35\textwidth}M{.2\textwidth}M{.35\textwidth}}
% 1. Zeile
	\multicolumn{2}{l}{Da dies kein Linksbaum ist,}
	&
	\multicolumn{1}{l}{ergibt sich}\\
% 2. Zeile
	\begin{tikzpicture}[sibling distance=8mm]
		\Tree [.\node[state](1){6};
			[.\node[EndPoint](2){}; ]
			[.\node[state](3){8}; 
				[.\node[EndPoint]{}; ]
				[.\node[EndPoint]{}; ]
			]
		]
		\node[node distance=5mm,right of=1]{1};
		\node[node distance=5mm,left of=2]{0};
		\node[node distance=5mm,right of=3]{1};
	\end{tikzpicture}
	&
	&
	\begin{tikzpicture}[sibling distance=5mm]
		\Tree [.\node[state]{6};
			[.\node[state]{8}; ]
			\edge[none];[.{} ]
		];
	\end{tikzpicture}
	\\
% 3. Zeile
	\multicolumn{2}{l}{\rule{0pt}{5ex}Nächster Schritt: Merge von}
	&
	\multicolumn{1}{l}{ergibt}\\
% 4. Zeile
	\begin{tikzpicture}[sibling distance=5mm]
		\Tree [.\node[state]{4};
			[.\node[state]{6};
				[.\node[state]{8}; ]
				[.\node[state]{6}; ]
			]
			\edge[none];[.{} ]
		];
	\end{tikzpicture}
	& 
	\begin{tikzpicture}[sibling distance=5mm]
		\Tree [.\node[state]{6};
			[.\node[state]{8}; ]
			\edge[none];[.{} ]
		];
	\end{tikzpicture}
	&
	\begin{tikzpicture}[sibling distance=5mm]
		\Tree [.\node[state]{4};
			[.\node[state]{6};
				[.\node[state]{8}; ]
				[.\node[state]{6}; ]
			]
			[.\node[state]{6};
				[.\node[state]{8}; ]
					\edge[none];[.{} ]
			]
		];
	\end{tikzpicture}	\\
% 5. Zeile
	\multicolumn{2}{l}{\rule{0pt}{5ex}Nächster Schritt: Merge von}
	&
	\multicolumn{1}{l}{ergibt}\\
% 6. Zeile
	\begin{tikzpicture}[sibling distance=5mm]
		\Tree [.\node[state]{4};
			[.\node[state]{6};
				[.\node[state]{8}; ]
				[.\node[state]{6}; ]
			]
			[.\node[state]{6};
				[.\node[state]{8}; ]
					\edge[none];[.{} ]
			]
		];
	\end{tikzpicture}
	&
	\begin{tikzpicture}[sibling distance=5mm]
		\Tree [.\node[state]{3};
			[.\node[state]{5};
				[.\node[state]{7}; ]
				\edge[none];[.{} ]
			]
			\edge[none];[.{} ]
		];
	\end{tikzpicture}
	&
	\begin{tikzpicture}[sibling distance=3mm]
		\Tree [.\node[state]{3};
			[.\node[state]{5};
				[.\node[state]{7}; ]
				\edge[none];[.{} ]
			]
			[.\node[state]{4};
				[.\node[state]{6};
					[.\node[state]{8}; ]
					[.\node[state]{6}; ]
				]
				[.\node[state]{6};
					[.\node[state]{8}; ]
						\edge[none];[.{} ]
				]
			]
		];
	\end{tikzpicture}	\\
% 7. Zeile
	\multicolumn{3}{l}{\rule{0pt}{5ex}{Da dieser Baum die Linksbaumeigenschaft verletzt muss er umgeformt werden}} \\
% 8. Zeile
	\begin{tikzpicture}[sibling distance=3mm]
		\Tree [.\node[state]{3};
			[.\node[state]{4};
				[.\node[state]{6};
					[.\node[state]{8}; ]
					[.\node[state]{6}; ]
				]
				[.\node[state]{6};
					[.\node[state]{8}; ]
						\edge[none];[.{} ]
				]
			]
			[.\node[state]{5};
				[.\node[state]{7}; ]
				\edge[none];[.{} ]
			]
		];
	\end{tikzpicture}
	&
	&
\end{tabular}

\newpage
\subsubsection{Laufzeitanalyse der Operation Merge}
Für die Laufzeitanalyse sind bei der Implentierung folgende Schritte notwendig:
\begin{itemize}
	\item Erzeugen der Teilbäume vor jeden Rekursionsschritt: \(\mathcal{O}(1)\)
	\item Zusammenbau der Teilbäume nach der Rekursion (nur wenn die s-Werte zwischengespeichert werden): \(\mathcal{O}(1)\)
	\item Anzahl der rekursiven Aufrufe: Die Rekursion endet, wenn der rechte Teilbaum nur aus einem Knoten besteht.
		Bei jedem rekursiven Aufruf verkleinert sich einer der rechten Teilbäume.
		Die Laufzeit beträgt daher \(\mathcal{O}(\log n)\)
\end{itemize}
Mit Hilfe der Merge Operation können alle anderen Heap-Operationen realisiert werden:
\begin{description}
	\item[put] merge(heap, \(<\)neuer Knoten\(>\))
	\item[removeMin] Wurzel enfernen und die zwei neuen Teilbäume mergen
	\item[init] Mit Hilfe von put in der Zeit \(\mathcal{O}(n \log n)\). Es gibt einen besseren Algorithmus für init welcher eine Laufzeit von \(\mathcal{O}(n)\) besitzt.
\end{description}

Mergesort kann man durch folgenden Algorithmus darstellen:
\begin{itemize}
	\item Max-Heap aufbauen: \(\mathcal{O}(n)\)
	\item n-mal die Funktion removeMax aufrufen und Elemente am Kopf einer Liste anfügen: \(\mathcal{O}(n \log n)\)
\end{itemize}
Damit ist die Laufzeit \(\mathcal{O}(n \log n)\), welche gleich der Average Laufzeit von Quicksort ist.
Es gibt ebenfalls eine In-Place-Implentierung die dem Heap als Array darstellt.
Damit hat HeapSort alle Vorteile von Quicksort und Mergesort ohne deren Nachteile zu besitzten.
\chapter{Dynamisches Programmieren}
Dynamische Programmierung ist eine Methode zum algorithmischen Lösen von Optimierungsproblemen.
Es kann dann erfolgreich eingesetzt werden, wenn das Optimierungsproblem aus vielen gleichartigen Teilproblemen besteht,
und eine optimale Lösung des Problems sich aus optimalen Lösungen der Teilprobleme zusammensetzt.
In der dynamischen Programmierung werden zuerst die optimalen Lösungen der kleinsten Teilprobleme direkt berechnet und dann geeignet zu einer Lösung eines nächstgrößeren Teilproblems zusammengesetzt.
Dieses Verfahren setzt man fort, bis das ursprüngliche Problem gelöst wurde.
Einmal berechnete Teilergebnisse werden in einer Tabelle gespeichert.
Bei nachfolgenden Berechnungen gleichartiger Teilprobleme wird auf diese Zwischenlösungen zurückgegriffen, anstatt sie jedes Mal neu zu berechnen.
Wird die dynamische Programmierung konsequent eingesetzt, vermeidet sie kostspielige Rekursionen, weil bekannte Teilergebnisse wiederverwendet werden.
Die Laufzeit eines dynamischen Programmieralgorithmuses ist Tabellengröße \(\cdot\) Aufwand pro Eintrag.

\subsubsection{Beispiel}
Ein Läufer nimmt in jeden Schritt 1 oder 2 Stufen auf einmal.
Wieviele Möglichkeiten gibt es, \(n\) Stufen herrauf zu steigen?
\begin{eqnarray*}
t_{1} &=& 1\\
t_{2} &=& 2\\
t_{n} &=& t_{n-1} + t_{n-2}
\end{eqnarray*}
Man kann zeigen das \(t_{n} \in \mathcal{O}(1,618^{n})\) liegt.
Wenn \(t_{n}\) rekursiv berechnet wird, werden die meisten \(t_{n}\) mehrfach berechnet, siehe Abbildung \ref{fig:ProgLaufer}.
Um \(t_{n}\) durch dynamische Programmierung zu berechnen, speichert man die Werte von \(t_{n}\) in einem Feld.
Die Laufzeit beträgt somit: \(\mathcal{O}(n)\cdot\mathcal{O}(1) = \mathcal{O}(n)\).
\newpage
\begin{figure}[htbp]
	\begin{center}
		\begin{tikzpicture}
		\Tree [
			.\(t_{n}\)
			[.\(t_{n-1}\)
				[.\(t_{n-2}\) ]
				[.\(t_{n-3}\) ]
			]
			[.\(t_{n-2}\)
				[.\(t_{n-3}\) ]
				[.\(t_{n-4}\) ]
			]
		]
		\end{tikzpicture}
	\end{center}
	\caption{Verdeutlichung der mehrfachen Berechnung von Werten}
	\label{fig:ProgLaufer}
\end{figure}
\begin{figure}[htbp]
\begin{lstlisting}[language=java, caption={Beispielimplementierung in Java}]
for(i=1; i<=n; i++) {
	if(n==1)
		t[n] = 1;
	else if(n==2)
		t[n] = 2;
	else
		t[n] = t[n-1] + t[n-2];
}
\end{lstlisting}
\end{figure}

\section{Editierdistanz}
Die Editierdistanz zwischen zwei Zeichenketten ist die minimale Anzahl von Einfüge-, Lösch- und Ersetz-Operationen, um die erste Zeichenkette in die zweite umzuwandeln.
In der Praxis wird die Editierdistanz zur Bestimmung der Ähnlichkeit von Zeichenketten beispielsweise zur Rechtschreibprüfung, DNA-Sequenzvergleich oder bei der Duplikaterkennung angewandt.

Gegeben seien zwei Strings a, b, wieviele Editieroperationen sind nötig, um a in b zu überführen?
\begin{figure}[htbp]
	\begin{center}
		\begin{tikzpicture}
			\node[cross] (1) {A};
			\node (2) [right of=1] {P};
			\node (3) [right of=2] {F};
			\node (4) [right of=3] {E};
			\node[cross] (5) [right of=4] {L};
			\node[red] (6) [right of=5] {D};
			\node[red] (7) [above of=5] {R};
			\node (8) [below of=2] {P};
			\node (9) [below of=3] {F};
			\node (10) [below of=4] {E};
			\node (11) [below of=5] {R};
			\node (12) [below of=6] {D};
		\end{tikzpicture}
	\end{center}
\end{figure}
Lösung: Die Editierdistanz beträgt 3.

Sei \(d(i,j)\) die Editierdistanz zwischen den Teilwörtern \(a_{1} \ldots a_{i}, b_{1} \ldots b_{j}\).
So gibt es folgende Möglichkeiten:
\begin{itemize}
	\item Ein Matching kann verlängert werden (MATCH)
		\begin{center}
			\begin{tikzpicture}[every node/.style={circle,inner sep=0pt}]
				\draw (0,1) -- (2,1) -- (2,1.5) -- (0,1.5);
				\draw (1.5,1) -- (1.5,1.5);
				\draw (1.75,1.2) node {\(a_{i}\)};
				\draw (0,0) -- (2,0) -- (2,.5) -- (0,.5);
				\draw (1.5,0) -- (1.5,.5);
				\draw (1.75,.2) node {\(b_{j}\)};
			\end{tikzpicture}
		\end{center}
		
		\(a_{i} = b_{j} \rightarrow\) Bewertung \(d(i-1,j-1)\)
	\item Missmatch
		\begin{center}
			\begin{tikzpicture}[every node/.style={circle,inner sep=0pt}]
				\draw (0,1) -- (2,1) -- (2,1.5) -- (0,1.5);
				\draw (1.5,1) -- (1.5,1.5);
				\draw (1.75,1.2) node {\(a_{i}\)};
				\draw (0,0) -- (2,0) -- (2,.5) -- (0,.5);
				\draw (1.5,0) -- (1.5,.5);
				\draw (1.75,.2) node {\(b_{j}\)};
			\end{tikzpicture}
		\end{center}
		
		\(a_{i} \neq b_{j} \rightarrow\) Bewertung \(d(i-1,j-1)+1\)
	\item Löschen / Hinzufügen
		\begin{center}
			\begin{tikzpicture}[every node/.style={circle,inner sep=0pt}]
				\draw (0,1) -- (2,1) -- (2,1.5) -- (0,1.5);
				\draw (1.5,1) -- (1.5,1.5);
				\draw (1.75,1.2) node {\(a_{i}\)};
				\draw (0,0) -- (1.5,0) -- (1.5,.5) -- (0,.5);
				\draw (1.25,.2) node {\(b_{j}\)};
			\end{tikzpicture}
		\end{center}
		
		Bewertung \(d(i-1,j)+1\)
		\begin{center}
			\begin{tikzpicture}[every node/.style={circle,inner sep=0pt}]
				\draw (0,1) -- (1.5,1) -- (1.5,1.5) -- (0,1.5);
				\draw (1.25,1.2) node {\(a_{i}\)};
				\draw (0,0) -- (2,0) -- (2,.5) -- (0,.5);
				\draw (1.5,0) -- (1.5,.5);
				\draw (1.75,.2) node {\(b_{j}\)};
			\end{tikzpicture}
		\end{center}
		
		Bewertung \(d(i,j-1)+1\)
\end{itemize}
Man wählt in jeden Schritt die Möglichkeit mit der besten Bewertung
\begin{equation*}
d(i,j) = \left\{
	\begin{array}{l l}
		j & \quad \text{für } i=0\\
		i & \quad \text{für } j=0\\
		\begin{array}{l l}
			\min \{&d(i-1,j-1) + 1_{a_{i} \neq b_{j}},\\
			&d(i-1,j) + 1,\\
			&d(i,j-1) + 1 \}
		\end{array} & \quad \text{für } i,j >0
  \end{array} \right.
\end{equation*}
wobei gilt:
\begin{equation*}
	  1_{a_{i} \neq b_{j}} = \left\{
	\begin{array}{l l}
		0 & \text{für } x = y \\
		1 & \text{für } x \neq y
	\end{array}
	\right.
\end{equation*}
Der Aufwand für die Berechnung in einem 2-dimensionalten Feld beträgt:
\begin{equation*}
	\underbrace{\mathcal{O}(n \cdot m)}_{\textrm{Größe der Tabelle}} \cdot \underbrace{\mathcal{O}(1)}_{\textrm{Vergleichsoperation}}
\end{equation*}
mit \(n = |a|, m= |b|\)

\subsubsection{Beispiel\protect\footnote{\url{https://youtu.be/qp8YwtvS3Uo}}}
Berechen Sie die Edierdistanz der Wörter: APFEL, PFERD.
\begin{center}
	\begin{tabular}{c|cccccc}
		& \(\varepsilon\) & A & P & F & E & L \\ \hline
	\(\varepsilon\)		& 0 & 1 & 2 & 3 & 4 & 5 \\
	P	& 1 & 1 & 1 & 2 & 3 & 4 \\
	F	& 2 & 2 & 2 & 1 & 2 & 3 \\
	E	& 3 & 3 & 3 & 2 & 1 & 2 \\
	R	& 4 & 4 & 4 & 3 & 2 & 2 \\
	D	& 5 & 5 & 5 & 4 & 3 & \textbf{\textcolor{red}{3}}
	\end{tabular}
\end{center}

\section{Längste gemeinsame Teilffolge}
Eine längste gemeinsame Teilffolge kann durch Streichen von Zeichen erzeugt werden.
Bespiel:
\begin{figure}[htbp]
	\begin{center}
		\begin{tikzpicture}
			\node (1) {A};
			\node (2) [right of=1] {N};
			\node (3) [right of=2] {A};
			\node (4) [right of=3] {N};
			\node[cross] (5) [right of=4] {A};
			\node[cross] (6) [right of=5] {S};
			\node (8) [below of=1] {A};
			\node (9) [below of=2] {N};
			\node (10) [below of=3] {A};
			\node (11) [below of=4] {N};
			\node[cross] (12) [below of=5] {E};
			\node[cross] (13) [left of=8] {B};
		\end{tikzpicture}
	\end{center}
\end{figure}
Die längste gemeinsame Teilffolge ist in diesem Bespiel 4.


Sei \(d(i,j)\) die Länge der längsten gemeinsamen Teilffolge von \(a_{1} \ldots a_{i}, b_{1} \ldots b_{j}\).
So gibt es folgende Möglichkeiten:

\begin{itemize}
	\item Teilffolge verlängern (sodass die letzten beiden Zeichen übereinstimmen)
		\begin{center}
			\begin{tikzpicture}[every node/.style={circle,inner sep=0pt}]
				\draw (0,1) -- (2,1) -- (2,1.5) -- (0,1.5);
				\draw (1.5,1) -- (1.5,1.5);
				\draw (1.75,1.2) node {\(a_{i}\)};
				\draw (0,0) -- (2,0) -- (2,.5) -- (0,.5);
				\draw (1.5,0) -- (1.5,.5);
				\draw (1.75,.2) node {\(b_{j}\)};
			\end{tikzpicture}
		\end{center}
		
		\(a_{i} = b_{j} \rightarrow\) Bewertung: \(d(i-1,j-1)+1\)
	\item Zeichen streichen
		\begin{center}
			\begin{tikzpicture}[every node/.style={circle,inner sep=0pt}]
				\draw (0,1) -- (2,1) -- (2,1.5) -- (0,1.5);
				\draw (1.75,1.2) node {\(a_{i}\)};
				\draw (0,0) -- (2,0) -- (2,.5) -- (0,.5);
				\draw (1.75,.2) node {\(b_{j}\)};
			\end{tikzpicture}
		\end{center}
		\(a_{i} \neq b_{j} \rightarrow\) Bewertung: \(d(i-1,j-1)\)
	\newpage
	\item Eines der letzten beiden Zeichen streichen
		\begin{center}
			\begin{tikzpicture}[every node/.style={circle,inner sep=0pt}]
				\draw (0,1) -- (2,1) -- (2,1.5) -- (0,1.5);
				\draw (1.5,1) -- (1.5,1.5);
				\draw (1.75,1.2) node[cross] {\(a_{i}\)};
				\draw (0,0) -- (1.5,0) -- (1.5,.5) -- (0,.5);
				\draw (1,0) -- (1,.5);
				\draw (1.25,.2) node {\(b_{j}\)};
			\end{tikzpicture}
		\end{center}
		Bewertung: \(d(i-1,j)\),\\ entsprechend andere Fall: \(d(i,j-1)\)
\end{itemize}
Damit gilt:
\begin{equation*}
d(i,j) = \left\{
	\begin{array}{l l}
		0 & \quad \text{für } i=0 \text{ oder } j=0\\
		\begin{array}{l l}
			\max \{&d(i-1,j-1) + 1_{a_{i} = b_{j}},\\
			&d(i-1,j),\\
			&d(i,j-1) \}
		\end{array} & \quad \text{für } i,j >0
  \end{array} \right.
\end{equation*}
Für die Laufzeit beträgt wie bei der Edierdistanz: \(\mathcal{O}(n \cdot m) \cdot \mathcal{O}(1) = \mathcal{O}(mn)\)

\subsubsection{Beispiel}
Berechen Sie die längste gemeinsame Teilfolge  der Wörter: BANANE, ANANAS.
\begin{center}
	\begin{tabular}{c|ccccccc}
		& \ & B & A & N & A & N & E \\ \hline
		& 0 & 0 & 0 & 0 & 0 & 0 & 0 \\
	A	& 0 & 0 & 1 & 1 & 1 & 1 & 1 \\
	N	& 0 & 0 & 1 & 2 & 2 & 2 & 2 \\
	A	& 0 & 0 & 1 & 2 & 3 & 3 & 3 \\
	N	& 0 & 0 & 1 & 2 & 3 & 4 & 4 \\
	A	& 0 & 0 & 1 & 2 & 3 & 4 & 4 \\
	S	& 0 & 0 & 1 & 2 & 3 & 4 & \textbf{\textcolor{red}{4}}
	\end{tabular}
\end{center}

\section{Komplexitätsklassen}
Eine Komplexitätsklasse bezeichnet eine Menge von Problemen, welche sich in einem ressourcenbeschränkten Berechnungsmodell berechnen lassen.
Definiert wird eine Komplexitätsklasse durch eine obere Schranke für den Bedarf einer bestimmten Ressource unter Voraussetzung eines Berechnungsmodells.
Die am häufigsten betrachteten Ressourcen sind die Anzahl der notwendigen Berechnungsschritte zur Lösung eines Problems oder der Bedarf an Speicherplatz.

Folgende Beispiele sollen die Einteilung verdeutlichen:
\begin{itemize}
	\item Rasenmähen hat mindestens lineare Komplexität (in der Fläche), denn man muss die gesamte Fläche mindestens einmal überfahren.
	\item Suchen im Telefonbuch hat hingegen nur logarithmische Komplexität, denn bei einem doppelt so dickem Telefonbuch schlägt man dieses nur einmal öfter auf (siehe Kapitel \ref{sec:binaereSuche})
\end{itemize}
Nachfolgend betrachten wir die die beiden Komplexitätsklassen P und NP genauer.

\begin{description}
	\item[P] enthält alle Probleme, die sich in der Zeit \(\mathcal{O}(n^{k})\) für ein \(k>0\) entscheiden lassen (polynomialer Zeit).
		Diese sind meist effizent lösbar.
		Beispiele:
			\begin{itemize}
				\item Sortieren (\(\mathcal{O}(n \log n)\)
				\item Editierdistanz (\(\mathcal{O}(m \cdot n) \subseteq \mathcal{O}((m+n)^{2})\)
			\end{itemize}
	\item[NP] enthält alle Probleme, die sich in der Zeit \(\mathcal{O}(n^{k})\) verifizieren (Prüfen einer gültigen Belegung) lassen.
		Diese sind nur exponentieller Zeit lösbar.
		Eine Unterklasse davon sind die NP-vollständigen Probleme.
		Diese sind mindestens so schwer wie alle Probleme in NP.
		Wichtige NP-vollständige Probleme:
		\begin{itemize}
			\item Erfüllbarkeitsproblem der Aussagenlogik (Gegeben sei eine Formel der Aussagenlogik. Ist diese erfüllbar?)
			\item Hamilton-Kreis (besitzt ein Graph einen Kreis, der jeden Knoten genau einmal besucht?)
			\item Travelling Salesmann Problem
			\item Rucksack-Problem
		\end{itemize}
\end{description}
Es gilt: Wenn ein NP-vollständiges Problem in P liegt, dann liegen alle NP-voll"-stän"-di"-gen Probleme in P.
Da bisher kein derartiges Problem gefunden wurde folgt daraus: Für kein NP-vollständiges Problem ist ein polynomialer Algorithmus bekannt.


\section{Travelling Salesmann Problem}
\label{sec:TSP}
Gesucht ist die kürzeste Rundreise durch \(n\) Städte, wobei jede Stadt genau einmal besucht wird.
Das TSP ist NP-vollständig.

Sei \((d_{ij})_{1 \leq i, j \leq n}\) die entsprechende Entfernungsmatrix.
Wir betrachten den allgemeinen Fall in dem \(d_{ij} \neq d_{ji}\) sowie \(d_{ij} = \infty\) gelten kann.
Der naive Algorithmus prüft alle \(n!\) Kombinationen und hat eine Laufzeit in \(\Omega(n!) \cdot \mathcal{O}(n) = \Omega(n!)\).

Optimalitätsprinzip: Wenn eine optimale Rundreise bei Stadt 1 beginnt und dann durch k führt, dann muss der Weg von k durch die Städte in \(\{2, \ldots, n\} - \{k\}\) ebenfalls optimal sein.
Sei \(l(i,S)\) die Länge des kürzesten Pfades, der bei \(i\) beginnt, dann durch jedes \(j \in S\) genau einmal führt und bei \(n\) endet.
\begin{figure}[tbp]
	\centering
	\begin{tikzpicture}[node distance=2cm]
		\draw (0,0) circle[radius=2pt] node[align=center, below] {i};
		\draw (2,0) circle[radius=2pt] node[align=center, below] {n};
		\draw[thick,decorate,decoration={brace,amplitude=12pt}] (2,-.5) -- (0,-.5) node[midway,below, yshift=-12pt]{s};
		\draw [zigzag] (.1,0) -- (1.9,0);
		\draw (4,0) circle[radius=2pt] node[align=center, below] {i};
		\draw (5,0) circle[radius=2pt] node[align=center, below] {j};
		\draw (7,0) circle[radius=2pt] node[align=center, below] {n};
		\draw[->] (4.1,0) -- (4.9,0) node[midway,below, yshift=-12pt]{\(d_{ij}\)};
		\draw [zigzag] (5.1,0) -- (6.9,0);
		\draw[thick,decorate,decoration={brace,amplitude=12pt}] (7,-.5) -- (5,-.5) node[midway,below, yshift=-12pt]{\(S-\{j\}\)};
	\end{tikzpicture}
	\caption{Veranschaulichung Optimalitätsprinzip}
\end{figure}
Die Länge des kürzesten Rundweges ist dann \(l(n, \{1, \ldots, n-1\})\).
Es gilt:
\begin{equation*}
l(i, S) = \left\{
	\begin{array}{l l}
		d_{in} & \quad \text{für } S = \varnothing \text{(keine Zwischenstädte)}\\
		\min\limits_{j \in S} \{d_{ij} + l(S-\{j\},j)\} & \quad \text{für alle } S \neq \varnothing
  \end{array} \right.
\end{equation*}

Mit Hilfe dieses Algorithmuses kann eine Matix erstellt werden, in welcher die Länge der Rundwege gepeichert werden.
In der x-Achse werden die Knoten von 1 bis \(n-1\) eingetragen und in der y-Achse die Teilmengen in aufsteigender Mächtigkeit.
Dieser besitzt eine Laufzweit von \(\mathcal{O}(2^{n-1} \cdot (n-1)) \cdot \mathcal{O}(n) = \mathcal{O}(n^{2} \cdot 2^{n})\).
Diese wächst langsamer als \(\mathcal{O}(n!)\).

Gegeben sei der nachfolgende Graph mit 4 Städten.
Gesucht ist der kürzeste Weg zur Stadt 4.
\begin{figure}[htbp]
	\begin{center}
		\begin{tikzpicture}[node distance=1.8cm]
			\node[state] (1) {1};
			\node[state] (2) [right of=1] {2};
			\node[state] (3) [below of=2] {3};
			\node[state] (4) [below of=1] {4};
			\path[-]
				(1) edge node[above] {2} (2)
					edge node[left, above] {1} (3)
					edge node[left] {2} (4)
				(2) edge node[right] {2} (3)
					edge node[right, below] {1} (4)
				(3) edge node[below] {2} (4);
		\end{tikzpicture}
	\end{center}
\end{figure}
\begin{center}
	\begin{tabular}{c|cccc}
	\{1,2,3\}	& X & X & X & Lösung: \begin{tabular}[x]{@{}c@{}c@{}}\(2+4=6\)\\\(1+5=6\)\\\(2+4=6\)\end{tabular}  \\ 
	\{2,3\}	& \begin{tabular}[x]{@{}c@{}}\(1+3=4\)\\\(2+4=6\)\end{tabular} & X & X & \\
	\{1,3\}	& X & \begin{tabular}[x]{@{}c@{}}\(2+3=5\)\\\(2+3=5\)\end{tabular} & X & \\
	\{1,2\}	& X & X & \begin{tabular}[x]{@{}c@{}}\(2+4=6\)\\\(1+3=4\)\end{tabular} & \\
	\{3\}	& \(2+2=3\) & \(2+2=4\) & X 		& \\
	\{2\}	& \(2+1=3\) & X 		& \(2+1=3\) & \\
	\{1\}	& X			& \(2+2=4\) & \(1+2=3\) & \\
	\{\(\varnothing\)\}		& 2 & 1 & 2 & \\ \hline
			& 1 & 2 & 3 & 4 
	\end{tabular}
\end{center}

\begin{shaded}
  \noindent
  \textbf{Def.:} Sei \(\varepsilon > 1\).
	Ein Minimierungsproblem heißt \(\varepsilon\)-approximierbar, wenn es einen Algorithmus mit polynomieller Laufzeit gibt, der eine Lösung liefert, die höchstems um \(\varepsilon\) größer ist als das Optimum.
\end{shaded}
Für \(P \neq NP\) ist das TSP für kein \(\varepsilon > 1\) approximierbar.
Falls die Entfernungsmatrix jedoch die Dreiecksungleichung \[d_{uv} \leq d_{uw} + d_{wv}\] gilt (\(\Delta\)-TSP), dann ist das Problem \(\frac{3}{2}\)-approximierbar.
Die Gültigkeit der Dreiecksungleichung lässt sich immer erreichen indem ggf. Kanten durch kürzere Wege ersetzt werden.

Ein Spezialfall des \(\Delta\)-TSP ist das Euklidischen TSP, bei dem die Abstände gleich dem geometrischen Abstand sind. Für jedes \(\delta > 1\) ist das Euklidische TSP in der Ebene \(1+\frac{1}{\delta}\)-approximierbar.
Der zugehörige Approximationsalgorithmus besitzt eine Laufzeit in \(\mathcal{O}(n \log(n))^{\mathcal{O}(\delta)}\).

Anwendungen des TSP:
\begin{itemize}
	\item Roboter soll Löcher in eine Platine bohren
	\item Auf einer Fertigungsstraße sollen Produkte \(P_{1}, \ldots, P_{n}\) herrgestellt werden.
		Dabei muss die Fertigungsstraße jeweils umgerüstet werden.
		Wenn \(d_{uv}\) die Zeit ist, die für das umrüsten von \(P_{U}\) nach \(P_{V}\) benötigt wird, muss ein TSP für (\(d_{uv}\)) gelöst werden..
		Für kleine \(n\) kann das exakte TSP und sonst das \(\Delta\)-TSP (Bedingung: Dreiecksungleichung)zur Approximation des Optimums verwendet werden.
	\item DNA-Sequenzierung (Erzeugung von DNA-Bruchstücken, Suche nach Über\-lap\-pungen, kürzester Pfad durch diese Knoten (Besser wäre allerdings der Aufbau einen Graphen und Suche eines euklidischen Kreises))
\end{itemize}



\section{Rucksackproblem}
Das Rucksackproblem ist ein Optimierungsproblem der Kombinatorik.
Aus einer Menge von Objekten, die jeweils ein Gewicht und einen Nutzwert haben, soll eine Teilmenge ausgewählt werden, deren Gesamtgewicht eine vorgegebene Gewichtsschranke nicht überschreitet.
Unter dieser Bedingung soll der Nutzwert der ausgewählten Objekte maximiert werden.
Anwendungen:
\begin{itemize}
	\item öffentliche Haushaltsführung
	\item Reduzierung des Verschnitts (Bsp. Fließen, Folien)
	\item Logistik (Bsp. Transport mittels Frachschiff)
\end{itemize}

Gegebenen seien n Gegenstände mit den Werten \(x_{1}, \dots, x_{n}\).
Gesucht ist eine Auswahl, die den Wert der Gegenstände maximiert und einen Schwellwert \(y\) nicht überschreitet.

Das Rucksackproblem ist NP-vollständig, es lässt sich mit einem dynamischen Pro\-gram\-mier-Algorithmus wie folgt lösen:
Sei \(r(n,y)\) der Wert einer Lösung des Rucksackproblems für die Werte \(x_{1}, \dots, x_{n}\) und der Rucksackgröße \(y\) dann gilt:
 \[r(n,y) = \left\{
			\begin{array}{l l}
				0 				& \quad \text{für } n=0\\
				r(n-1,y) 		& \quad \text{für } n>0 \wedge x_{n} > y \\
				\begin{array}{l l}
					\max \{
						r(n-1,y),\\
						r(n-1,y-x_{n})+x_{n}
					\}
				\end{array} & \quad \text{sonst}
		\end{array} \right.\]

Die Laufzeit des Rucksack-Algorithmuses beträgt: \(\mathcal{O}(ny)\).
Es stellt sich daher die Frage ob dies ein polynomieller Algorithmus für das Rucksackproblem ist (woraus P=NP folgen würde)?

Probleme werden in der Komplexitätstheorie als Mengen dargestellt und die Laufzeit als Funktion in der Länge einer Instanz.
Die Länge einer Instanz ist \((x_1, \dots, x_n, y)\).
Um \(y\) über dem Alphabet \({0,1}\) (binär) darzustellen werden \(\log_2y + \mathcal{O}(1)\) Zeichen benötigt.
Als Funktion der Länge der Eingabe ergibt sich für die Laufzeit \(ny = n2^{\log_2y}\).
Die Laufzeit ist damit exponentiell in der Länge der Eingabe.

\begin{shaded}
	\noindent
	\textbf{Def.:} Ein Algorithmus heißt pseudopolynomiell, wenn seine Laufzeit durch ein Polynom in der Eingabelänge und der größten, in der Eingabe vorkomenden Zahl beschränkt ist.
\end{shaded}
Das Rucksackproblem ist pseudopolynomiell: Sein \(|w|\) die Eingabelänge und \(m = \max(x_1, \dots, x_n, y)\) (Das Längste Wort der Kodierung).
Dann gilt: \(n\leq |w|\), woraus \(\mathcal{O}(ny) \subseteq \mathcal{O}(|w|m)\) folgt.
\chapter{Graphalgorithmen}
Aus der Vorlesung Künstliche Intelligenz sind bereits die Algorithmen Breiten- und Tiefensuche bekannt.
\begin{lstlisting}[language=java, caption={Beispiel Algorithmus für die Breitensuche}]
boolean bfs(start, goal) {

	// Anfangs sind keine Knoten besucht
	for(v in V)
		discovered[v] = false;
	
	// Mit Start-Knoten beginnen
	queue.enqueue(start)
	discovered[start] = true;
	
	while(!queue.isEmpty()){
		// Erstes Element von der queue nehmen
		u = queue.dequeue;
		
		// Testen ob Zielknoten gefunden
		if(u == goal)
			return true;

		// alle Nachfolge-Knoten, ...
		for(v in adj[u]))
			// ... die noch nicht besucht wurden ...
			if(!discovered[v]){
				// ... zur queue hinzufuegen ...
				queue.enqueue(v);
				// ... und als bereits gesehen markieren
				discovered[u] = true;
			}
	}
	return false;
}
\end{lstlisting}
\newpage
Für dieses Beispiel fallen folgende Laufzeiten an:
\begin{itemize}
	\item Zeile 4-5: \(\mathcal{O}(|V|)\)
	\item Zeile 8-9: \(\mathcal{O}(1)\)
	\item Zeile 13-17: \(\mathcal{O}(1)\)
	\item Zeile 20-27: \(\mathcal{O}(\deg(u))\)
	\item Zeile 29: \(\mathcal{O}(1)\)
\end{itemize}
\(\deg(u)\) steht dabei für den Grad des Knotens (=Anzahl der Nachbarn).
Ein Knoten \(v\) hat den Grad \(k\) wenn \(v\) mit genau \(k\) anderen Knoten verbunden ist.
Wir schreiben dafür \(\deg(u) = k\).

\begin{shaded}
	\noindent
	\textbf{Satz:} Die Laufzeit der Breitensuche liegt in \(\mathcal{O}(|V|+|E|)\).

	Dabei ist \(|V|\) die Anzahl der Knoten (Vertex) und \(|E|\) die Anzahl der Kanten (Edge) im Graphen.
\end{shaded}

\paragraph{Beweis:}
Das initialisieren des Feldes discovered benötigt die Zeit \(\mathcal{O}(|V|)\).
Um die unbesuchten Nachbarn des Knotens \(u\) zu bestimmen fällt der Aufwand \(\mathcal{O}(\deg(u))\) an.
Da jeder Knoten höchstens einmal aus der Warteschlange entnommen wird, wird auch die while Schleife für jeden Knoten höchstens einmal durchlaufen.
Der gesamte Aufwand ist damit
\[\mathcal{O}(|V|) + \sum\limits_{u \in V} \mathcal{O}(\deg(u) = \mathcal{O}(|V|) + \mathcal{O}(|E|) = \mathcal{O}(|V| + |E|)\]
Die Tiefensuche lässt sich implentieren wie die Breitensuche, wenn anstelle der Warteschlange ein Stack verwendet wird.
Die Laufzeit ist gleich der Breitensuche: \(\mathcal{O}(|V| + |E|)\).


\section{Verallgemeinerung der A*-Suche}
Hierbei wird eine heuristische Bewertungsfunktion \(f\) verwendet um Knoten einzusortieren.
Die heuristische Bewertungsfunktion hat die Gestalt
\[f(v) = g(v) + h(v)\]
wobei
\begin{itemize}
	\item \(g(v)\) die Kosten bis zum Knoten v
	\item \(h(v)\) eine zulässige Kostenschätzfunktion
\end{itemize}
sind.
Eine Kostenschätzfunktion \(h\) ist zulässig, wenn sie die Kosten zum Ziel nicht überschätzt.
\begin{figure}[htbp]
	\centering
	\begin{tikzpicture}[node distance=2cm]
		\draw (0,0) circle[radius=4pt] node[align=center, below, yshift=-4mm] {Start};
		\draw (2,0) circle[radius=4pt] node[align=center, below, yshift=-4mm] {v};
		\draw [zigzag] (0.2,0) -- (1.8,0);
		\draw (4,0) circle[radius=4pt] node[align=center, below, yshift=-4mm] {Ziel};
		\draw [dotted] (2.2,0) -- (3.8,0);
		\draw[thick,decorate,decoration={brace,amplitude=12pt}] (2,-.8) -- (0,-.8) node[midway,below, yshift=-12pt]{\(g(v)\)};
		\draw[thick,decorate,decoration={brace,amplitude=12pt}] (4,-.8) -- (2,-.8) node[midway,below, yshift=-12pt]{\(h(v)\)};
	\end{tikzpicture}
	\caption{Veranschaulichung der heuristische Bewertungsfunktion}
\end{figure}
\newpage
Bei einer Navigation könnten die Funktionen wie folgt definiert sein:
\begin{itemize}
	\item \(g(v)\): Strecke von Start bis v
	\item \(h(v)\): Luftlinienentfernung von v zum Ziel
\end{itemize}

\subsubsection{Übung}
Für ein Streckennetz sind Entfernungen und durchschnittliche Geschwindigkeiten bekannt.
Wie kann die schnellste Route von A nach B gefunden werden?
\begin{itemize}
	\item \(g(v)\): Zeit für Strecke von Start bis v (\(t=\frac{s}{v}\))
	\item \(h(v)\): \(t=\frac{\textrm{Luftlinie bis zum Ziel}}{\textrm{maximale Geschwindigkeit der verbleibenden Kanten}}\)
\end{itemize}


\section{Topologisches Sortieren}
\begin{shaded}
  \noindent
  \textbf{Def.:} Ein DAG (Directed acyclic graph) ist ein gerichteter Graph, der keine gerichteten Kreise enthält.
		Eine topologische Sortierung eines DAG \(G = (V,E)\) ist eine Abbildung
		\[f: V \rightarrow \mathbb{N} \textrm{ mit } f(u) < f(v) \textrm{ für } (u,v) \in E\]
\end{shaded}
\begin{figure}[htbp]
	\begin{center}
	 	\begin{tikzpicture}[sibling distance=5mm]
			\node[state] (1) at (1,0) {1};
			\node[state] (2) at (0,-1) {2};
			\node[state] (3) at (2,-1) {4};
			\node[state] (4) at (-1,-2) {4};
			\draw[->] (1) -- (2);
			\draw[->] (1) -- (3);
			\draw[->] (2) -- (3);
			\draw[->] (2) -- (4);
		\end{tikzpicture}
		\caption{Beispiel für ein Directed acyclic graph (DAG)}
	\end{center}
\end{figure}

Jeder vollständige Graph oder Kreis lässt sich nicht topologisch sortieren.
Eine topologische Sortierung kann durch eine Tiefensuche bestimmt werden.
\newpage
\begin{lstlisting}[language=java,caption={Pseudocode für eine topologische Sortierung}]
topsort:
	for (v in V)
		markiere v mit weiss
	for (v in V)
		tiefensuche(v)

tiefensuche(v):
	v grau:
		Fehler("Kreis vorhanden")
	v weiss:
		markiere v mit grau
		for (u in adj[v])
			tiefensuche(u)
		markiere v mit schwarz
		fuege v an den Kopf einer Liste
\end{lstlisting}

\begin{shaded}
	\noindent
	\textbf{Satz:} Für jeden DAG \(G=(V,E)\) erzeugt Topsort eine topologische Sortierung von \(G\)
\end{shaded}

\paragraph{Beweis:}
Sei \((u,v) \in E\).
In u und in v werden je eine Tiefensuche gestartet.
Die in v gestartete Tiefensuche endet früher als die in u gestartete Tiefensuche.
Daher wird u links von v in die Liste eingefügt und erhält daher eine kleinere Nummer als v.
\chapter{Datenkompression}
\section{Huffman Codierung}
\subsubsection{Idee:}
Häufige Zeichen erhalten kurze Codewörter, seltene Zeichen längere Codewörter.
Gesucht ist ein Code, so dass die mittlere Codewortlänge minimal ist.
\subsubsection{Problem:}
Die Codierung muss eindeutig decodierbar sein.

\begin{shaded}
  \noindent
  \textbf{Def.:} Ein Präfixcode ist ein Code, so dass kein Codewort Präfix eines anderen Codewortes ist.
\end{shaded}

\subsubsection{Beispiel}
\begin{center}
	\begin{tabular}{lcccccc}
		Zeichen & a & b & c & d & e & f\\
		Wahrscheinlichkeit & 0,45 & 0,13 & 0,12 & 0,16 & 0,09 & 0,05\\
		Code & 0 & 101 & 100 & 111 & 1101 & 1100
	\end{tabular}
\end{center}
Dieser Code ist optimal.
In Abbildung \ref{fig:HuffmanTree} ist der Code als Codewortbaum dargestellt.
\begin{figure}[htbp]
	\centering
	\begin{tikzpicture}[sibling distance=5mm]
		\tikzstyle{end} = [circle, minimum width=4pt,fill, inner sep=0pt]
		\Tree
		[.\node {};
			\edge node[auto=right] {0};
			[.a ]
			\edge node[auto=left] {1};
			[.\node[end] {};
				\edge node[auto=right] {0};
				[.\node[end] {};
					\edge node[auto=right] {0};
					[.c ]
					\edge node[auto=left] {1};
					[.b ]
				]
				\edge node[auto=left] {1};
				[.\node[end] {};
					\edge node[auto=right] {0};
					[.\node[end] {};
						\edge node[auto=right] {0};
						[.f ]
						\edge node[auto=left] {1};
						[.e ]
					]
					\edge node[auto=left] {1};
					[.d ]
				]
			]
		]
	\end{tikzpicture}
	\caption{Die Folge 1001010 lässt sich eindeutig decodieren zu cba}
	\label{fig:HuffmanTree}
\end{figure}
Der Huffman-Algorithmus ist ein Greedy-Algorithmus (wählt den nächsten Schritt nach der besten Wahrscheinlichkeit) der einen optimalen Präfix-Code konstruiert.

\subsection{Implentierung}
\begin{lstlisting}[language=java]
Initialisierung: Jedes Zeichen ist ein Baum mit einem Knoten.

while (mehr als ein Baum vorhanden)
	Verbinde zwei Baeume mit den beiden niedrigsten
	Wahrscheinlichkeiten zu einem neuen Baum,
	die Wahrscheinlichkeiteb addieren sich.
\end{lstlisting}
\subsubsection{Beispiel}
\begin{figure}[htbp]
	\centering
	\begin{tikzpicture}[sibling distance=5mm]
		\node[state] 				(1)		{f};
		\node[above of=1] 			(11)	{0,05};
		\node[state, right of=1]	(2)		{e};
		\node[above of=2] 			(21)	{0,09};
		\node[state, right of=2]	(3)		{c};
		\node[above of=3] 			(31)	{0,12};
		\node[state, right of=3]	(4)		{b};
		\node[above of=4] 			(41)	{0,13};
		\node[state, right of=4]	(5)		{d};
		\node[above of=5] 			(51)	{0,16};
		\node[state, right of=5]	(6)		{a};
		\node[above of=6] 			(61)	{0,45};
	\end{tikzpicture}
\end{figure}

\begin{center}
	\begin{tabular}{llll}
		% 1. Zeile
		\begin{tikzpicture}[sibling distance=5mm]
			\node[state]				(3)		{c};
			\node[above of=3] 			(11)	{0,12};
			\node[state, right of=3]	(4)		{b};
			\node[above of=4] 			(11)	{0,13};
		\end{tikzpicture}
		&
		\multicolumn{2}{c}{
			\begin{tikzpicture}[sibling distance=10mm]
				\tikzstyle{end} = [circle, minimum width=4pt,fill, inner sep=0pt]
				\Tree
				[.\node {0,14};
					\edge node[auto=right] {0};
					[.f ]
					\edge node[auto=left] {1};
					[.e ]
				]
			\end{tikzpicture}
		}
		&
		\begin{tikzpicture}[sibling distance=5mm]
			\node[state]	(5)		{d};
			\node[above of=5] 			(11)	{0,16};
			\node[state, right of=5]	(6)		{a};
			\node[above of=6] 			(11)	{0,45};
		\end{tikzpicture}
		\vspace{0.5cm}
		\\ 
		% 2. Zeile
		\begin{tikzpicture}[sibling distance=10mm]
			\tikzstyle{end} = [circle, minimum width=4pt,fill, inner sep=0pt]
			\Tree
			[.\node {0,14};
				\edge node[auto=right] {0};
				[.f ]
				\edge node[auto=left] {1};
				[.e ]
			]
		\end{tikzpicture}
		& 
		\begin{tikzpicture}[sibling distance=5mm]
			\node[state]				(3)		{d};
			\node[above of=3] 			(31)	{0,16};
		\end{tikzpicture}
		& 
		\begin{tikzpicture}[sibling distance=10mm]
			\tikzstyle{end} = [circle, minimum width=4pt,fill, inner sep=0pt]
			\Tree
			[.\node {0,25};
				\edge node[auto=right] {0};
				[.c ]
				\edge node[auto=left] {1};
				[.b ]
			]
		\end{tikzpicture}
		&
		\begin{tikzpicture}[sibling distance=5mm]
			\node[state]				(3)		{a};
			\node[above of=3] 			(31)	{0,45};
		\end{tikzpicture}
		\vspace{0.5cm}
		\\
		% 3. Zeile
		\begin{tikzpicture}[sibling distance=10mm]
			\tikzstyle{end} = [circle, minimum width=4pt,fill, inner sep=0pt]
			\Tree
			[.\node {0,25};
				\edge node[auto=right] {0};
				[.c ]
				\edge node[auto=left] {1};
				[.b ]
			]
		\end{tikzpicture}
		&
		\multicolumn{2}{c}{
			\begin{tikzpicture}[sibling distance=5mm]
				\tikzstyle{end} = [circle, minimum width=4pt,fill, inner sep=0pt]
				\Tree
				[.\node {0,3};
					\edge node[auto=right] {0};
					[ \edge node[auto=right] {0};
						[.f ]
						\edge node[auto=left] {1};
						[.e ]
					]
					\edge node[auto=left] {1};
					[.d ]
				]
			\end{tikzpicture}
		}
		&
		\begin{tikzpicture}[sibling distance=5mm]
			\node[state]				(3)		{a};
			\node[above of=3] 			(31)	{0,45};
		\end{tikzpicture}
		\vspace{0.5cm}
		\\
		% 4. Zeile
		\begin{tikzpicture}[sibling distance=5mm]
			\node[state]				(3)		{a};
			\node[above of=3] 			(31)	{0,45};
		\end{tikzpicture}
		&
		\multicolumn{3}{c}{
			\begin{tikzpicture}[sibling distance=5mm]
				\tikzstyle{end} = [circle, minimum width=4pt,fill, inner sep=0pt]
				\Tree
				[.\node {0,55};
					\edge node[auto=right] {0};
					[
						\edge node[auto=right] {0};
						[.c ]
						\edge node[auto=left] {1};
						[.b ]
					]
					\edge node[auto=left] {1};
					[
						\edge node[auto=right] {0};
						[
							\edge node[auto=right] {0};
							[.f ]
							\edge node[auto=left] {1};
							[.e ]
						]
						\edge node[auto=left] {1};
						[.d ] ]
				]
			\end{tikzpicture}
		}
	\end{tabular}
\end{center}
\begin{center}
	\begin{tikzpicture}[sibling distance=5mm]
		\tikzstyle{end} = [circle, minimum width=4pt,fill, inner sep=0pt]
		\Tree
		[.\node {};
			\edge node[auto=right] {0};
			[.a ]
			\edge node[auto=left] {1};
			[
				\edge node[auto=right] {0};
				[
					\edge node[auto=right] {0};
					[.c ]
					\edge node[auto=left] {1};
					[.b ]
				]
				\edge node[auto=left] {1};
				[
					\edge node[auto=right] {0};
					[
						\edge node[auto=right] {0};
						[.f ]
						\edge node[auto=left] {1};
						[.e ]
					]
					\edge node[auto=left] {1};
					[.d ] ]
			]
		]
	\end{tikzpicture}
\end{center}

\chapter{Lernverfahren}
\section{Entscheidungsbäume}
Entscheidungsbäume dienen der Klassifizierung von Daten.
Die Inneren Knoten sind dabei die Attribute und die Blätter stellen die Zielvariablen da.
\begin{figure}[htbp]
	\centering
	\begin{tikzpicture}
		[every node/.style={fill=black!10,rounded corners,align=center},
		grow=south, level distance=2cm,
		level 1/.style={sibling distance=9cm},
		level 2/.style={sibling distance=5cm},
		]

		\node{Entfernung}
			child{
				node{ja} edge from parent node[draw=green]{\(\leq 100\)}}
			child{
				node{Wochenende}
					child{
						node {Sonne}
							child{
								node {ja}edge from parent node[draw=green]{ja}}
							child{
								node {Nein} edge from parent node[draw=green]{nein}}
						edge from parent node[draw=green]{ja}}
					child{
						node {Nein} edge from parent node[draw=green]{nein}}
					edge from parent node[draw=green]{\(\geq 100\)}}
		% es dürfen keine Leerzeilen dazwischen sein
		;
	\end{tikzpicture}
	\caption{Entscheidungsbaum für die Klassifizierung ``Fahren wir Ski?''}
	\label{fig:EntscheidungsbaumBsp}
\end{figure}
\begin{table}[htbp]
	\centering
	\begin{tabular}{ccccc}
		\toprule
		\textbf{Nr.} & \textbf{Entfernung} & \textbf{Wochenende} & \textbf{Sonne} & \textbf{Ski}	\\
		\midrule
		1 & \(\leq 100\) & j & j & j\\
		2 & \(\leq 100\) & j & j & j\\
		3 & \(\leq 100\) & j & n & j\\
		4 & \(\leq 100\) & n & j & j\\
		5 & \(>100\) & j & j & j\\
		6 & \(>100\) & j & j & j\\
		7 & \(>100\) & j & j & n\\
		8 & \(>100\) & j & n & n\\
		9 & \(>100\) & n & j & n\\
		10 & \(>100\) & n & j & n\\
		11 & \(>100\) & n & n & n	\\
		\bottomrule
	\end{tabular}
	\caption{Trainingsdaten für Entscheidungsbaum in Abbildung \ref{fig:EntscheidungsbaumBsp}}
\end{table}

Da sich die Daten in Nr. 6, 7 wiedersprechen, können nicht alle Daten richtig klassifiziert werden.

Um den Entscheidungsbaum aus den Trainingsdaten aufzubauen, wird der Informationsgewinn (Informationstheoretischer Wert) der Attribute berechnet.
Der Wurzelknoten unterscheidet nach dem Attribut mit dem höchsten Informationsgewinn.
Auf die dadurch entstandenen Daten wird das Verfahren rekursiv angewendet, d.h. die anhand des ersten Knotens unterteilten Daten werden durch das Attribut mit dem höchsten Informationsgewinn weiter unterteilt.
Das Verfahren endet, wenn es keine Attribute mehr gibt oder der verbleibende Informationsgewinn 0 ist.

Wir betrachten die Trainingsdaten als Realisierung von unabhängigen Zufallsvariablen \(A_1,\dots, A_k\) (Attribute) und einer davon abhängigen Zufallsvariable \(y\) (Zielgröße).

Eigenschaften der Entropie:
\begin{itemize}
	\item Entropie ist maximal bei Gleichverteilung (\(\log_2 n\))
	\item Entropie ist 0, wenn die Verteilung nur einen Wert annimmt.
\end{itemize}
\subsubsection{Beispiel}
Sei \(y\) gleichverteilt auf \(\{1, \ldots, 6\}\)
\begin{itemize}
	\item \(H(Y) = \log_2 6\)
	\item \(H(Y|Y gerade) = \log_2 3\)
	\item \(H(Y|Y=6) = \log_2 1 = 0\)
\end{itemize}
Der Informationsgewinn für y bei beobachteten A ist:
\begin{eqnarray*}
 G(Y,A) &=& H(Y) - E H(Y|A)\\
		&=& H(Y) - \sum\limits_{a \in A(\Omega)} P(A=a) \cdot H(Y|A=a)
\end{eqnarray*}
Wegen \(0 \leq H(Y|A=a) \leq H(Y)\) ist \(0 \leq G(Y|A) \leq H(Y)\)

\newpage
\subsubsection{Übung}
Berechnen Sie \(Y = \) Skifahren.
\begin{itemize}
	\item \(H(Y) = -(\frac{6}{11} \cdot \log_2\frac{6}{11} + \frac{5}{11} \cdot \log_2\frac{5}{11}) = 0,994\)
	\item \(G(Y|\textrm{Entfernung}) = H(Y) - E H(Y|E)\)
			\begin{eqnarray*}
			 	H(Y|E \leq 100) &=& 0\\
			 	H(Y|E > 100) &=& -(\frac{2}{7} \cdot \log_2\frac{2}{7} + \frac{5}{7} \cdot \log_2\frac{5}{7}) = 0,863\\
			 	\curvearrowright G(Y|E) &=& 0,994 - (\frac{4}{11} \cdot 0 + \frac{7}{11} \cdot 0,863)\\
			 	&=& 0,445
			\end{eqnarray*}
	\item \(G(Y|\textrm{Wochenende}) = H(Y) - E H(Y|W)\)
			\begin{eqnarray*}
			 	H(Y|W = \textrm{ja}) &=& -(\frac{5}{7} \cdot \log_2\frac{5}{7} + \frac{2}{7} \cdot \log_2\frac{2}{7}) = 0,863\\
			 	H(Y|W = \textrm{nein}) &=& -(\frac{1}{4} \cdot \log_2\frac{1}{4} + \frac{3}{4} \cdot \log_2\frac{3}{4}) = 0,811\\
			 	\curvearrowright G(Y|W) &=& 0,994 - (\frac{7}{11} \cdot 0,863 + \frac{4}{11} \cdot 0,811)\\
			 	&=& 0,149 \approx 0,15
			\end{eqnarray*}
	\item \(G(Y|\textrm{Sonne}) = H(Y) - E H(Y|S)\)
			\begin{eqnarray*}
			 	H(Y|S = \textrm{ja}) &=& -(\frac{5}{8} \cdot \log_2\frac{5}{8} + \frac{3}{8} \cdot \log_2\frac{3}{8}) = 0,954\\
			 	H(Y|S = \textrm{nein}) &=& -(\frac{1}{3} \cdot \log_2\frac{1}{3} + \frac{2}{3} \cdot \log_2\frac{3}{3}) = 0,918\\
			 	\curvearrowright G(Y|S) &=& 0,994 - (\frac{8}{11} \cdot 0,954 + \frac{3}{11} \cdot 0,918)\\
			 	&=& 0,049
			\end{eqnarray*}
\end{itemize}

Da das Attribut Entfernung den größten Informationsgewinn für die Zielgröße Y besitzt, wird die Entfernung zum Unterscheidungskriterium an der Wurzel des Entscheidungsbaums.
\begin{figure}[htbp]
	\centering
	\begin{tikzpicture}
		[every node/.style={fill=black!10,rounded corners,align=center},
		grow=south, level distance=2cm,
		level 1/.style={sibling distance=5cm},
		level 2/.style={sibling distance=2cm},
		]

		\node{Entfernung}
			child{
				node{\(\dots\)} edge from parent node[draw=green]{\(\leq 100\)}}
			child{
				node{\(\dots\)} edge from parent node[draw=green]{\(\geq 100\)}}
		% es dürfen keine Leerzeilen dazwischen sein
			;
	\end{tikzpicture}
\end{figure}

Da \(H(Y|E \leq 100) = 0\), muss diese Datenmenge nicht weiter untereilt werden.
Jedoch ist \(H(Y|E > 100) > 0\).
Für die Daten, für die \(E \> 100\) gilt, wird das Verfahren rekursiv fortgeführt, bis alle Attribute verwendet sind oder der verbleibende Informationsgewinn 0 ist.
Es ergibt sich der Entscheidungsbaum aus Abbildung \ref{fig:EntscheidungsbaumBsp}.

Vorteile von Entscheidungsbäumen:
\begin{itemize}
	\item Kriterien des Entscheidungsbaums sind nachvollziehbar, keine Blackbox (Im Gegensatz zu neuronalen Netzen). 
	\item Wichtigkeit der Kriterien anhand der Position im Entscheidungsbaum erkennbar.
		Interessant für Marktforschung und verkleinern des Entscheidungsbaums, fals er schlecht generalisiert.
\end{itemize}
Nachteile von Entscheidungsbäumen:
\begin{itemize}
	\item Der Algorithmus kann nicht erkennen, ob ein Attribut mit hoher Entropie sinnvoll ist, z.B. Kreditkartennummer.
	\item Stetige Attribute müssen diskretisiert werden.
\end{itemize}

\subsubsection{Übung}
Gegeben sei nachfolgenden Trainigsdaten von Pilzen.
Erstellen Sie einen Entscheidungsbaum für das Attribut Essbar.
\begin{table}[htbp]
	\centering
	\begin{tabular}{cccc}
		\toprule
		\textbf{Farbe} & \textbf{Größe} & \textbf{Punkte} & \textbf{Essbar}	\\
		\midrule
		rot & klein & ja & nein\\
		braun & klein & nein & ja\\
		braun & groß & ja & ja\\
		grün & klein & nein & ja\\
		rot & groß & nein & ja	\\
		\bottomrule
	\end{tabular}
\end{table}

\begin{eqnarray*}
	H(Y)	&=& -(\frac{1}{5} \cdot \log_2(\frac{1}{5}) + \frac{4}{5} \cdot \log_2(\frac{4}{5}))	\\
			&=& 0,722\\\\
	G(Y,F)	&=& H(Y) - E H(Y|F)\\
			&=& 0,722 - (\frac{2}{5} \cdot 1 + \frac{2}{5} \cdot 0 + \frac{1}{5} \cdot 0)			\\
			&=& 0,322\\
	H(Y|F=\textrm{rot})		&=& \log_2 2 = 1\\
	H(Y|F=\textrm{braun})	&=& \log_2 1 = 0\\
	H(Y|F=\textrm{grün})	&=& \log_2 1 = 0
\end{eqnarray*}
\begin{eqnarray*}
	G(Y,G)	&=& H(Y) - E H(Y|G)\\
			&=& 0,722 - (\frac{3}{5} \cdot 0,918 + \frac{2}{5} \cdot 0)			\\
			&=& 0,171\\
	H(Y|G=\textrm{klein})	&=& -(\frac{2}{3} \cdot \log_2(\frac{2}{3}) + \frac{1}{3}\cdot \log_2(\frac{1}{3})) = 0,918\\	
	H(Y|G=\textrm{groß})	&=& \log_2 1 = 0	\\
	\\
	G(Y,P)	&=& H(Y) - E H(Y|P)\\
			&=& 0,722 - (\frac{2}{5} \cdot 1 + \frac{3}{5} \cdot 0)			\\
			&=& 0,322\\
	H(Y|P=\textrm{ja})	&=& \log_2 2 = 1\\
	H(Y|P=\textrm{nein})	&=& \log_2 1 = 0
\end{eqnarray*}
Das Attribut Punkte hat den größten Informationsgewinn und steht somit an oberster Stelle.
Es wird jetzt rekursiv die nächsten Ebenen berechnet
\begin{eqnarray*}
	H(Y|P=\textrm{nein})		&=& -(\frac{3}{3} \cdot \log_2 \frac{3}{3})	\\
								&=& 0
\end{eqnarray*}
Da die Entropie für das Attribut keine Punkte gleich 0 ist, muss dieses nicht weiter untereilt werden.
\begin{eqnarray*}
	H(Y|P=\textrm{ja})			&=& -(\frac{1}{2} \cdot \log_2 \frac{1}{2})	\\
								&=& 1	\\
	G(Y|F,P=\textrm{ja})		&=& H(Y) - E H(Y|F)\\
								&=& 1 - (\frac{1}{2} \cdot 0 + \frac{1}{2} \cdot 0)	\\
								&=& 1\\
	H(Y|F=\textrm{rot}, P=\textrm{ja})		&=& 1 \cdot \log_2 1 = 0\\
	H(Y|F=\textrm{braun}, P=\textrm{ja})	&=& 1 \cdot \log_2 1 = 0\\
	\\
	G(Y|G,P=\textrm{ja})		&=& H(Y) - E H(Y|F)\\
								&=& 1 - (\frac{1}{2} \cdot 0 + \frac{1}{2} \cdot 0)	\\
								&=& 1\\
	H(Y|G=\textrm{groß}, P=\textrm{ja})		&=& 1 \cdot \log_2 1 = 0\\
	H(Y|G=\textrm{klein}, P=\textrm{ja})	&=& 1 \cdot \log_2 1 = 0\\
\end{eqnarray*}
Da der Informationsgewinn für das Attribut Farbe und Größe gleich ist, ist die Auswahl egal.
Wir wählen als nächstes Attribut Punkte.
\begin{eqnarray*}
	H(Y|P=\textrm{ja}, G=\textrm{klein})	&=& -(\frac{1}{1} \cdot \log_2 \frac{1}{1})	\\
								&=& 0	\\
	H(Y|P=\textrm{ja}, G=\textrm{groß})		&=& -(\frac{1}{1} \cdot \log_2 \frac{1}{1})	\\
								&=& 0
\end{eqnarray*}
Da die Entropie für beide Ausprägungen 0 ist, endet hier die Rekursion.
Eine weitere Einteilung würde keinen Informationsgewinn bringen
\begin{figure}[htbp]
	\centering
	\begin{tikzpicture}
		[every node/.style={fill=black!10,rounded corners,align=center},
		grow=south, level distance=2cm,
		level 1/.style={sibling distance=9cm},
		level 2/.style={sibling distance=5cm},
		]

		\node{Punkte}
			child{
				node{Farbe}
					child{
						node {nein} edge from parent node[draw=green]{rot}}
					child{
						node {ja} edge from parent node[draw=green]{braun}}
					edge from parent node[draw=green]{ja}}
			child{
				node{ja} edge from parent node[draw=green]{nein}}
		% es dürfen keine Leerzeilen dazwischen sein
		;
	\end{tikzpicture}
	\caption{Entscheidungsbaum für die Klassifizierung ``Pilze essbar?''}
\end{figure}
\section{Markov-Ketten, Hidden Markov Modelle}
Eine Markov-Kette ist ein stochastischer Prozess, der durch Zustände und Über-gangswahrscheinlichkeiten beschrieben wird.
\begin{figure}[htbp]
	\centering
		\begin{tikzpicture}[node distance=2cm]
			\node[state] (z_0)                {$z_0$};
			\node[state] (z_1) [right of=z_0] {$z_1$};
			\path[->] 
				(z_0) 
					edge [loop above] node {0,3} ()
					edge [bend left, above] node {0,7} (z_1)
				(z_1) 
					edge [loop above] node {0,6} ()
					edge [bend left, below] node {0,4} (z_0);
		\end{tikzpicture}
\end{figure}

Damit lassen sich modellieren:
\begin{itemize}
	\item Texte in natürlicher Sprache (Zustände: Buchstaben)
	\item DNA-Sequenzen (Zustände: A,C,G,T)
	\item Navigation im Internet (Zustände: Webseiten)
\end{itemize}


\subsection{Hidden Markov Model (HMM)}
\begin{shaded}
  \noindent
  \textbf{Def.:} Ein HHM ist eine Markov-Kette, die in jeden Zustand z ein Zeichen a ausgibt mit der Wahrscheinlichkeit \(q_z(a)\).
\end{shaded}
\begin{figure}[htbp]
	\centering
	\begin{tikzpicture}[node distance=4cm]
		\node[state] (z_0)                {\(z_0\)};
		\node[state] (z_1) [right=4cm of z_0] {\(z_1\)};
		\node[emission] (a) [below of=z_0] {a};
		\node[emission] (b) [below of=z_1] {b};
		\path[->] 
			(z_0) 
				edge [loop above] node {0,5} ()
				edge [bend left, above] node {0,5} (z_1)
				edge [left] node {0,6} (a)
				edge [left] node {0,4} (b)
			(z_1) 
				edge [loop above] node {0,9} ()
				edge [ above] node {0,1} (z_0)
				edge [right] node {0,9} (b)
				edge [right] node {0,1} (a);
	\end{tikzpicture}
\end{figure}
Beobachtet werden nur die vom HMM ausgegebenen Zeichen.
Unbekannt ist die Zustandsfolge.
Mit HMM können modelliert werden z.B.:
\begin{itemize}
	\item Codierende Regionen in DNA\\
		Zustände: A,C,T,G in codierenden und nicht codierenden Regionen\\
		Ausgegebene Zeichen: Jeweils A,C,T,G
	\item Nachrichtenübertragung\\
		Markov-Kette, die Texte modelliert\\
		In einem Zustand z wird das Zeichen z ausgegeben mit großer Wahrscheinlichkeit (z.B. 0,95), andere Zeichen mit geringer W'keit.
	\item Spracherkennung\\
		Vorgehen: Audio-Signal $\to$ FFT $\to$ Phoneme $\to$ Wort\\
		Zustände: Phoneme, die der Sprecher gesprochen hat.\\
		Ausgegebenen Zeichen: Phoneme, die das System erkannt hat oder das vom System erkannte Wort.
\end{itemize}

\subsubsection{Mathematische Behandlung}
Sei \(Z_k\) eine Zufallsvariable, die den Zustand des HMM im Schritt \(k\) angibt.
Aus \(Z_1\) ergibt sich die Startverteilung \[\pi_k=P(Z_1=k)\] der Markov-Kette.
Da \(Z_{k+1}\) nur von \(Z_k\) abhängt, folgt:
\begin{eqnarray*}
	&&P(\underbrace{Z_{k+1}}_{\text{Zufallsvariable}}=\underbrace{Z_{i_{k+1}}}_{\text{Zustand}}
    \mid Z_k =z_{i_k}, \ldots, Z_1=z_{i_1})\\
	&=& P(Z_{k+1} = z_{i_{k+1}}\mid Z_k=z_{i_k})\\
	&=:& p(Z_{i_k},Z_{i_{k+1}})
\end{eqnarray*}
Sei \(A_k\) die Zufallsvariable, die das in Schritt \(k\) ausgegebene Zeichen angibt.
\(A_k\) hängt nur von \(Z_k\) ab:
\[P(A_k=a\mid Z_k=z_{i_k}, \ldots, Z_1=z_1) = P(A_k=a\mid Z_k=z_{i_k}) := q_{z_{i_k}}(a)\]
Die charakteristischen Größen eines HMM sind damit \(\pi_k,\; p(z,z'),\; q_z(a)\).

\subsubsection{Rekonstruktion der Zustandsfolge}
Für eine Folge \(a_1,\dots,a_n\) von beobachteten Zeichen suchen wir eine Folge \(z_{i_1}, \ldots, z_{i_n}\) von Zuständen, so dass
\[P(Z_1=z_{i_1}, \ldots, Z_n=z_{i_n} \mid  A_1=a_1, \ldots, A_n=a_n)\]
maximal ist.
Da in
\begin{eqnarray*}
    \frac{P(Z_1=z_{i_1}, \dots, Z_n=z_{i_n},\; A_1 = a_1, \dots, A_n=a_n)}{P(A_1=a_1, \dots, A_n=a_n)} \\
\end{eqnarray*}
der Nenner unabhängig von der Lösung ist (da er der Beobachtung entspricht), maximieren wir den Zähler.
%\begin{eqnarray*}
	%&&P(Z_1=z_1, \dots, Z_n=z_n \mid  A_1=a_1, \dots A_n=a_n)\\
	%&=& \frac{P(Z_1=z_1, \dots, Z_n=z_n) \cap P(A_1 = a_1, \dots, A_n=a_n)}{P(A_1=a_1, \dots, A_n=a_n)} \\
	%&=& \frac{P(Z_1=z_1, \dots, Z_n=z_n, A_1 = a_1, \dots, A_n=a_n)}{P(A_1=a_1, \dots, A_n=a_n)}
%\end{eqnarray*}

Sei
\begin{eqnarray*}
	t(z_{i_n},n) &=& P(Z_1=z_{i_1}, \dots, Z_n=z_{i_n}, A_1=a_1, \dots, A_n=a_n)\\
		&=& P(A_n=a_n\mid Z_1=z_{i_1}, \dots, Z_n=z_{i_n}, A_1=a_1, \dots, A_{n-1} = a_{n-1}) \cdot \\
		&& P(Z_1=z_{i_1}, \dots, Z_n=z_{i_n}, A_1=a_1, \dots, A_{n-1}=a_{n-1})\\
		&=& P(A_n=a_n\mid Z_n=z_{i_n}) \cdot \\
		&& P(Z_1=z_{i_1}, \dots, Z_n=z_{i_n}, A_1=a_1, \dots, A_{n-1}=a_{n-1})\\
		&=& q_{z_{i_n}}(a_n) \cdot\\
		&& P(Z_n=z_{i_n}\mid Z_1=z_{i_1}, \dots, Z_{n-1}=z_{i_{n-1}}, A_1=a_1, \dots, A_{n-1}=a_{n-1}) \cdot\\
		&& P(Z_1=z_{i_1},\dots, Z_{n-1}=z_{i_{n-1}}, A_1=a_1, \dots, A_{n-1}=a_{n-1})\\
		&=& q_{z_{i_n}}(a_n) \cdot P(Z_n=z_{i_n}\mid Z_{n-1}=z_{i_{n-1}}) \cdot \\
		&& P(Z_1=z_{i_1}, \dots, Z_{n-1}=z_{i_{n-1}}, A_1=a_1, \dots, A_{n-1}=a_{n-1})\\
		&=& q_{z_{i_n}}(a_n) \cdot p(z_{i_{n-1}},z_{i_n}) \cdot t(z_{i_{n-1}},n-1)
\end{eqnarray*}



\subsection{Viterbi-Algorithmus}
Der Viterbi-Algorithmus ist ein Algorithmus der dynamischen Programmierung zur Bestimmung der wahrscheinlichsten Sequenz
von verborgenen Zuständen bei einem gegebenen Hidden Markov Model (HMM) und einer beobachteten Sequenz von Symbolen.
Diese Zustandssequenz wird auch als Viterbi-Pfad bezeichnet.
\[
    t(Z,n) =
    \begin{cases}
        q_Z(a_1) \cdot \pi_Z 								& \text{für } n=1\\
        q_Z(a_n) \cdot \displaystyle \max_{z'}\left(p(z',z) \cdot t(z',n-1)\right) & \text{für } n>1\\
    \end{cases}
\]
Die Laufzeit beträgt \(\underbrace{\mathcal{O}(|Z| \cdot n)}_{\text{Größe der Tabelle}} \cdot \underbrace{\mathcal{O}(|Z|)}_{\text{Aufwand pro Zelle}} = \mathcal{O}(|Z|^2 \cdot n)\)
\begin{figure}[htbp]
	\centering
	\begin{tikzpicture}[node distance=1.5cm]
		\node[state] (z)                  {z};
		\node[state] (z') [left=1.5cm of z]     {z'};
		\node[state] (e1) [above of=z']  {};
		\node[state] (e2) [below of=z'] {};
		\node (a) [below=.5cm of e2] {\(\underbrace{a_1\; \dots\; a_{k-1}}_{t(z',k-1)}\)};
		\node[emission] (ak) [below of=z] {$a_k$};
		\path[->] 
			(e1) edge (z)
			(e2) edge (z)
			(z') edge [above] node {P(z',z)} (z)
			(z)	 edge [right] node {\(q_z(a_k)\)} (ak);
	\end{tikzpicture}
    \caption{Verdeutlichung des Viterbi-Algorithmusses.}
    \label{fig:viterbi}
\end{figure}

Um die numerische Stabilität des Verfahrens zu erhöhen, kann man mit Logarithmen der Werte rechnen.

\subsubsection{Beispiel}
Gegeben sei das HMM-Model in Abbildung \ref{fig:hmmbsp} mit der Folge ``k z z k k z k k k k k k'' (mit k=Kopf, z=Zahl).
Berechnen Sie die wahrscheinlichste Zustandsfolge die diese Folge erzeugt hat
\begin{figure}[htbp]
	\centering
	\begin{tikzpicture}[node distance=2cm]
		\node[state] (z0)                 {z0};
		\node[state] (z1) [right of=z0]   {z1};
		\node[emission] (k) [below of=z0] {k};
		\node[emission] (z) [below of=z1] {z};
		\path[->] 
			(0.9,1.5) 
				edge [left] node {0,9} (z0)
				edge [right] node {0,1} (z1)
			(z0) 
				edge [loop left] node {0,9} ()
				edge [bend left, above] node {0,1} (z1)
				edge [left] node {0,5} (k)
				edge [left] node {0,5} (z)
			(z1) 
				edge [loop right] node {0,9} ()
				edge [bend left, above] node {0,1} (z0)
				edge [right] node {0,1} (z)
				edge [right] node {0,9} (k);
	\end{tikzpicture}
    \caption{HMM für Beispiel X.}
    \label{fig:hmmbsp}
\end{figure}

\begin{center}
\begin{tabular}{cccc}
				& \(z_{0}\) & \(z_{1}\) & \\ \hline
\(t(z',1)\)= k	& \(0,5\cdot0,9 = 0,4500;\) & \(0,9\cdot0,1 = 0,0900\) & \(z_{0}\) \\ \hline
\(t(z',2)\)= z	& \(	\begin{array} {r@{}l@{}}
							0,5\cdot\max\{	& 0,9\cdot0,45; \\
											& 0,1\cdot0,09\}\\
										   =& 0,2025
					\end{array}
					\)
				&  \(	\begin{array} {r@{}l@{}}
							0,1\cdot\max\{	& 0,9\cdot0,09; \\
											& 0,1\cdot0,45\}\\
										   =& 0,0081
					\end{array}
					\) & \(z_{0}\) \\ \hline
\(t(z',3)\)= z	& \(	\begin{array} {r@{}l@{}}
							0,5\cdot\max\{	& 0,9\cdot0,2; \\
											& 0,1\cdot0,0081\}\\
										   =& 0,091125
					\end{array}
					\)
				&   \(	\begin{array} {r@{}l@{}}
							0,1\cdot\max\{	& 0,9\cdot0,008; \\
											& 0,1\cdot0,2\}\\
										   =& 0,002025
					\end{array}
					\) & \(z_{0}\)\\ \hline
\(t(z',4)\)= k	& \(	\begin{array} {r@{}l@{}}
							0,5\cdot\max\{	& 0,9\cdot0,09; \\
											& 0,1\cdot0,002\}\\
										   =& 0,04100625
					\end{array}
					\)
				&   \(	\begin{array} {r@{}l@{}}
							0,9\cdot\max\{	& 0,9\cdot0,002; \\
											& 0,1\cdot0,09\}\\
										   =& 0,00820125
					\end{array}
					\) & \(z_{0}\)\\ \hline
\(t(z',5)\)= k	& \(	\begin{array} {r@{}l@{}}
							0,5\cdot\max\{	& 0,9\cdot0,041; \\
											& 0,1\cdot0,0082\}\\
										   =& 0,0184528125
					\end{array}
					\)
				&   \(	\begin{array} {r@{}l@{}}
							0,9\cdot\max\{	& 0,9\cdot0,0082; \\
											& 0,1\cdot0,041\}\\
										   =& 0,0066430125
					\end{array}
					\) & \(z_{0}\)\\ \hline
\(t(z',6)\)= z	& \(	\begin{array} {r@{}l@{}}
							0,5\cdot\max\{	& 0,9\cdot0,018; \\
											& 0,1\cdot0,007\}\\
										   =& 0,0083037656
					\end{array}
					\)
				&   \(	\begin{array} {r@{}l@{}}
							0,1\cdot\max\{	& 0,9\cdot0,007; \\
											& 0,1\cdot0,018\}\\
										   =& 0,00063
					\end{array}
					\) & \(z_{0}\)\\ \hline
\(t(z',7)\)= k	& \(	\begin{array} {r@{}l@{}}
							0,5\cdot\max\{	& 0,9\cdot0,0081; \\
											& 0,1\cdot0,00063\}\\
										   =& 0,0037366945
					\end{array}
					\)
				&   \(	\begin{array} {r@{}l@{}}
							0,9\cdot\max\{	& 0,9\cdot0,00063; \\
											& 0,1\cdot0,0083\}\\
										   =& 0,0007473389
					\end{array}
					\) & \(z_{1}\)\\ \hline
\(t(z',8)\)= k	& \(	\begin{array} {r@{}l@{}}
							0,5\cdot\max\{	& 0,9\cdot0,0037; \\
											& 0,1\cdot0,00075\}\\
										   =& 0,0016815254
					\end{array}
					\)
				&   \(	\begin{array} {r@{}l@{}}
							0,9\cdot\max\{	& 0,9\cdot0,00075; \\
											& 0,1\cdot0,0037\}\\
										   =& 0,0006053445
					\end{array}
					\) & \(z_{1}\)\\ \hline
\(t(z',9)\)= k	& \(	\begin{array} {r@{}l@{}}
							0,5\cdot\max\{	& 0,9\cdot0,00162; \\
											& 0,1\cdot0,00059\}\\
										   =& 0,0007566806
					\end{array}
					\)
				&   \(	\begin{array} {r@{}l@{}}
							0,9\cdot\max\{	& 0,9\cdot0,0006; \\
											& 0,1\cdot0,00162\}\\
										   =& 0,0004903291
					\end{array}
					\) & \(z_{1}\)\\ \hline
\(t(z',10)\)= k	& \(	\begin{array} {r@{}l@{}}
							0,5\cdot\max\{	& 0,9\cdot0,000729; \\
											& 0,1\cdot0,0004779\}\\
										   =& 0,000328
					\end{array}
					\)
				&   \(	\begin{array} {r@{}l@{}}
							0,9\cdot\max\{	& 0,9\cdot0,0004779; \\
											& 0,1\cdot0,000729\}\\
										   =& 0,0003971665
					\end{array}
					\) & \(z_{1}\)\\ \hline
\(t(z',11)\)= k	& \(	\begin{array} {r@{}l@{}}
							0,5\cdot\max\{	& 0,9\cdot0,000328; \\
											& 0,1\cdot0,0006561\}\\
										   =& 0,0001476
					\end{array}
					\)
				&   \(	\begin{array} {r@{}l@{}}
							0,9\cdot\max\{	& 0,9\cdot0,0006561; \\
											& 0,1\cdot0,000328\}\\
										   =& 0,0005314
					\end{array}
					\) & \(z_{1}\)\\ \hline
\(t(z',12)\)= k	& \(	\begin{array} {r@{}l@{}}
							0,5\cdot\max\{	& 0,9\cdot0,0001476; \\
											& 0,1\cdot0,0005314\}\\
										   =& 0,00006642
					\end{array}
					\)
				&   \(	\begin{array} {r@{}l@{}}
							0,9\cdot\max\{	& 0,9\cdot0,0005314; \\
											& 0,1\cdot0,001476\}\\
										   =& 0,0004304
					\end{array}
					\) & \(z_{1}\)
\end{tabular}
\end{center}
Damit ergibt sich die Zustandsfolge \(z_{0} \rightarrow z_{0} \rightarrow z_{0} \rightarrow z_{0} \rightarrow z_{1} \rightarrow z_{1}\rightarrow z_{1} \rightarrow z_{1}\rightarrow z_{1}\rightarrow z_{1}\)

Ausgehend vom maximum des letzten Zustandes ist der Vorgänger der Zustand, aus dem das Maximum hervorging.

\subsection{Parameterschätzung}
Die Parameter eines HMM sind Startverteilung, Übergangs- und Emissionswahrscheinlichkeiten.
Gegeben sei eine Menge von Trainingssequenzen.

\subsubsection{Überwachtes Lernen}
Wenn für alle Trainingssequenzen die Zustandsfolge bekannt ist, lassen sich ML-Schät"-zer (Maximum-Likelihood) für alle Parameter angeben.
Entsprechende Trainingssequenzen lassen sich häufig erzeugen, z.B.
\begin{itemize}
	\item Nachrichtenübertragung: Nachricht mehrfach senden und empfangen.
	\item Sprachverarbeitung: Beispielsätze für die darin enthaltenen Phoneme bekannt sind, werden vorgelesen.
\end{itemize}
Seien \(z,z\)' Zustände des HMM und \(h(z,z')\) die Häufigkeit des Übergangs von \(z\) nach \(z'\) in den Trainingssequenzen.
Sei ferner \(h_0(z)\) die Häufigkeit von \(z\) als Startzustand.
Da der Übergang von \(z\) nach \(z\)' durch eine Bernoulli-Verteilung beschrieben werden kann, sind
\[
	\hat{p}(z,z')=\frac{h(z,z')}{\sum\limits_{z''} h(z,z'')} \qquad
	\hat\pi_z=\frac{h_0(z)}{\|Z\|}
\]
ML-Schätzer für \(p(z,z')\) bzw \(\pi_z\).
Entsprechend ist
\[
	\hat{q}_z(a)=\frac{h_z(a)}{\sum\limits_{a'} h_z(a')}
\]
ein ML-Schätzer für \(q_z(a)\), wobei \(h_z(a)\) die Häufigkeit der Emission des Zeichens \(a\) im Zustand \(z\) in den Trainingssequenzen ist.
Wenn im HMM gilt: \(\sum_a q_z(a)=1\), ist \(\sum_{a'} h_z(a')\) die Summe der Längen aller Trainingssequenzen.

\subsubsection{Unüberwachtes Lernen}
Wenn die Zustandsfolge für die Trainingssequenzen nicht bekannt ist, können die unbekannten Parameter durch ein iteratives Verfahren geschätzt werden.
Idee dazu:
\begin{itemize}
	\item Mit zufälligen Parametern beginnen oder alle Wahrscheinlichkeiten gleich setzen.
	\item Aus den Trainingssequenzen mit dem Viterbi-Algorithmus die Zustandsfolge rekonstruieren, damit die Schätzwerte für \(h_0(z),\; h(z,z'),\; h_z(a)\) berechnen.
	\item Mit den oben beschriebenen Verfahren Schätzer für die Parameter des HMM berechnen.
\end{itemize}
Die letzten beiden Schritte werden wiederholt, bis ein Terminierungskriterium erreicht wird.

\subsubsection{Viterbi-Training}
\begin{algorithm}
	\begin{algorithmic}[1]
		\State {Parameter \(p(z,z'),\; \pi_z,\; q_z(a)\) zufällig oder durch überwachtes Lernen initialisieren.}
		\Repeat
		\State{Wende den Viterbi-Algorithmus auf die Trainingssequenzen an}
		\State{Berechne \(\hat{p}(z,z'),\; \hat\pi_z,\; \hat{q_z}(a)\)}
		\Until{keine Änderung an \(\hat{p}(z,z'),\; \hat\pi_z,\; \hat{q_z}(a)\).}
	\end{algorithmic}
	\caption{Unüberwachtes Lernen für HMM.}
	\label{alg:hmmus}
\end{algorithm}
Der Algorithmus \ref{alg:hmmus} terminiert, weil schließlich der Viterbi-Algorithmus stets die gleiche Folge liefert (ohne Beweis) und die geschätzten Parameter sich daher nicht mehr ändern.
Das Viterbi-Training liefert jedoch keinen ML-Schätzer für die unbekannten Parameter.
Ein besserer Algorithmus ist der Baum-Welch-Algorithmus (auch Ex\-pect\-ation-Maximization-Algorithmus genannt).
Dieser findet ein lokales Maximum der Likelihood-Funktion:
	\[P(A_1=a_1, \dots, A_n=a_n\mid Theta)\]
wobei \(a_1, \dots, a_n\) die beobachtete Ausgabe und \(\Theta\) die Menge der zu schätzenden Parameter des HMM ist.

\subsection{Forward-Algorithmus}
\begin{shaded}
	\noindent
	\textbf{Wdh. Disjunkte Zerlegung:}
	\begin{eqnarray*}
		P(A) = P(A \cap \Omega) &=& P(A \cap (B \cup \bar{B}))\\
								&=& P((A \cap B) \cup (A \cap \bar{B}))\\
								&=& P(A \cap B + A \cap \bar{B})\\
								&=& P(A \cap B)+P(A \cap \bar{B})
	\end{eqnarray*}
	\begin{center}
		\begin{tikzpicture}
			\draw (0,0) rectangle (3,2);
			\draw (1.8,0) -- (1.8,2);
			\draw (1.3,1.0) ellipse (1.0cm and 0.7cm);
			\draw (0.2,0.2) node {$B$};
			\draw (2.8,0.2) node {$\bar{B}$};
			\draw (1.3,0.7) node {$A$};
			\draw (3.2,1.8) node {$\Omega$};
		\end{tikzpicture}
	\end{center}
	
    \begin{eqnarray*}
        P(A) = P(A \cap \Omega) &=& P(A \cap \sum\limits^n_{k=1} B_k)\\
        &=& P(\sum\limits_{k=1}^n(A \cup B_k))\\
                                &=& P((A \cap B) \cup (A \cap \bar{B}))\\
                                &=& P(A \cap B + A \cap \bar{B})\\
                                &=& P(A \cap B)+P(A \cap \bar{B})
    \end{eqnarray*}
    \begin{center}
		\begin{tikzpicture}
			\draw (0,0) rectangle (3,2);
			\draw (.6,0) -- (.6,2);
			\draw (1.2,0) -- (1.2,2);
			\draw (1.8,0) -- (1.8,2);
			\draw (2.4,0) -- (2.4,2);
			\draw (1.3,1.0) ellipse (1.0cm and 0.7cm);
			\draw (0.2,-.3) node {\(B_1\)};
			\draw (0.8,-.3) node {\(B_2\)};
			\draw (2.8,-.3) node {\(B_n\)};
			\draw (1.3,0.7) node {$A$};
			\draw (3.2,1.8) node {$\Omega$};
		\end{tikzpicture}
	\end{center}
	\(P(A) = \sum_i P(A \cap B_i)\), wenn \(B_i \Omega\) partitioniert.

\end{shaded}
Für eine Beobachtung \(a_1, \dots, a_n\) eines HMM suchen wir \(P(A_1=a_1, \dots, A_n=a_n)\).
Die Schwierigkeit dabei ist, dass die Zustandsfolge unbekannt ist.
Der naiver Ansatz ist die disjunkte Zerlegung nach Zustandsfolge \(Z_1, \dots, Z_n\):
\[P(A_1=a_1, \dots, A_n=a_n) = \sum\limits_{z_1, \dots, z_n} P(A_1=a_1, \dots, A_n=a_n, Z_1=z_1, \dots, Z_n=z_n)\]
Da diese Summe aus \(|Z|^n\) Summanden besteht, ist dies jedoch ineffizient.
Die Laufzeit wäre: $O(|Z|^n\cdot n)$.

Mit dynamischer Programmiereung erhalten wir ein effizenters Verfahren dazu sei:
\begin{eqnarray}
	\alpha_t(j) &=& P(A_1=a_1, \dots, A_t=a_t,Z_t=j)\nonumber\\
				&=& \sum_i P(A_1=a_1, \dots, A_t=a_t, Z_{t-1}=i, Z_t=j)		\nonumber\\
				&=& \sum_i P(A_t=a_t\mid A_1=a_1, \dots, _{t-1}=a_{t-1}, Z_{t-1}=i,Z_t=j) \nonumber\\
				&& \cdot P(A_1=a_1, \dots, A_{t-1}=a_{t-1}, Z_{t-1}=i, Z_t=j)\nonumber\\
				&=& \sum_i P(A_t=a_t\mid Z_t=j) \cdot P(A_1=a_1,\dots, A_{t-1}=a_{t-1},Z_{t-1}=i,Z_t=j)\nonumber\\
				&=& \sum_i q_j(a_t) \cdot P(Z_t=j\mid Z_{t-1}=i) \cdot P(A_1=a_1, \cdots, A_{t-1}=a_{t-1}, Z_{t-1}=i)\nonumber\\
				&=& \sum_i q_j(a_t) \cdot p(i,j) \cdot \alpha_{t-1}(i)
				\label{eq:Forward_1}
\end{eqnarray}
Für t=1 gilt
\begin{equation}
	\alpha_1(j)=P(A_1=a_1, Z_1=j)=q_j(a_1)\cdot \pi_j
	\label{eq:Forward_2}
\end{equation}
Ferner gilt
\begin{eqnarray}
	P(A_1=a_1, \dots,A_n=a_n)	&=&\sum P(A_1=a_1, \dots, A_n=a_n, Z_n=j) \nonumber\\
								&=& \sum_j \alpha_n (j)
	\label{eq:Forward_3}
\end{eqnarray}
Mit den Ergebnissen aus (\ref{eq:Forward_1}), (\ref{eq:Forward_2}) und (\ref{eq:Forward_3}) lässt sich die Wahrscheinlichkeit der Beobachtung \(a_1, \dots, a_n\) berechnen.
Aufwand dazu \[\underbrace{\mathcal{O}(n\cdot|Z|)}_{\text{Tabellengröße}}\cdot\underbrace{\mathcal{O}(|Z|)}_{\text{Aufwand pro Zelle}}+\underbrace{\mathcal{O}(|Z|)}_{\text{Aufsummierung}}=\mathcal{O}(n\cdot|Z|^2)\]

\begin{figure}[htbp]
	\centering
	\begin{tikzpicture}[node distance=1.5cm]
		\node[state] (j)                 {j};
		\node[state] (i) [left=2cm of j]	{i};
        \node[state] (s1) [above=of i]	{};
		\node[state] (s2) [below=of i]	{};
		\node (at-1) [below =.5cm of s2] {\(a_{t-1}\)};
        \node (dots) [left=of at-1] {\(\dots\)};
        \node (a1) [left=of dots] {$a_1$};
		%\node[emission] (at) [below of=j] {\(a_t\)};
        \node[emission] (at) [right = 1.85cm of at-1] {\(a_t\)};
        \path[->]
			(s1) edge (j)
			(s2) edge (j)
			(i) edge [above] node {\(p(i,j)\)} (j)
			(j)	edge [right] node {\(q_j(a_t)\)} (at);
	\end{tikzpicture}
	\caption{Skizze des Forward-Algorithmus}
	\label{fig:Forward-Alg}
\end{figure}

\subsection{Backward-Algorithmus}
Gegeben eine Beobachtung \(a_1, \dots, a_n\), was ist der wahrscheinlichste Zustand in Schritt \(t\)?
Gesucht ist ein \(j\) mit \[P(Z_t=j\mid A_1=a_1, \dots, A_n=a_n)\] maximal.

Ansatz:
\begin{eqnarray*}
	&& P(A_1=a_1, \dots, A_n=a_n,Z_t=j)\\
	&=& P(A_{t+1}=a_{t+1}, \dots, A_n=a_n\mid A_1=a_1, \dots, A_t=a_t, Z_t=j)\\
	&&\cdot P(A_1=a_1, \dots, A_t=a_t,Z_t=j)\\
	&=& P(A_{t+1}=a_{t+1}, \dots, A_n=a_n\mid Z_t=j)\cdot P(A_1=a_1,\dots, A_t=a_t, Z_t=j)\\
	&=& \beta_t(j)\cdot\alpha_t(j)
\end{eqnarray*}
Ferner sei \(\beta_n(j)=1\) für alle \(j\).
Ähnlich wie in Forward-Algorithmus folgt:
\[\beta_t(j)=\sum_i p(j,i)\cdot q_i(a_{t+1}) \cdot \beta_{t+1}(i)\]

\begin{figure}[htbp]
	\centering
	\begin{tikzpicture}[node distance=1.5cm]
		\node[state] (j)				{\(j\)};
		\node[state] (s1)	[right of=j]	{};
		\node[state] (s2)	[above of=s1]	{};
		\node[state] (i)	[below of=s1]	{\(i\)};
		\node[emission] (at+1)	[below of=i]	{\(a_{t+1}\)};
		\node[emission] (at)	[left of=at+1]	{\(a_{t}\)};
		\node (...)				[right of=at+1]	{\(\dots\)};
		\node[emission]	(...)	[right of=...]	{\(a_n\)};
		\path[->] 
			(j) edge (s1)
			(j) edge (s2)
			(j) edge (i)
			(j) edge [right] node {\(p(j,i)\)} (i)
			(i)	edge [right] node {\(q_i(a_{t+1})\)} (at+1);
	\end{tikzpicture}
	\caption{Skizze des Backward-Algorithmuses}
\end{figure}

Die Laufzeit des Backward-Algorithmus liegt damit in
\[\underbrace{\mathcal{O}(n\cdot|Z|^2)}_{\alpha\text{-Tabelle}} + \underbrace{\mathcal{O}(n\cdot|Z|^2)}_{\beta\text{-Tabelle}} + \underbrace{\mathcal{O}(|Z|)}_{\text{Summe}}+\underbrace{\mathcal{O}(1)}_{\text{Zugriff auf }\alpha_t(j) \text{und }\beta_t(j)}=\mathcal{O}(n\cdot|Z|^2)\]


Die Wahrscheinlichkeit für den Zustand \(j\) im Schritt \(t\) ergibt sich aus
\begin{align*}
    P(Z_t=j\mid A_1=a_1, \dots, A_n=a_n) =&
    \frac{P(A_1=a_1,\ldots,A_n=a_n,Z_t=j)}{P(A_1=a_1,\ldots,A_n=a_n)}\\
=&\frac{\alpha_t(j)\cdot\beta_t(j)}{\sum_i \alpha_n(i)}
\end{align*}


\subsection{Posterior Decoding}
Wenn der Viterbi-Algorithmus viele unterschiedliche Pfade mit annähernd gleicher Wahrscheinlichkeit liefert, dann lässt sich die Wahl des wahrscheinlichsten Pfades nicht gut rechtfertigen.
Alternativ können wir mit dem Backward-Algorithmus die Folge der in jedem
Zeitpunkt $t$ wahrscheinlichsten Zustände
\[ \hat z_{t} = \arg_{j}\max P(Z_{t} = j \mid  A_{1} = a_{1}, \ldots, A_{n} = a_{n}) \]
bestimmen.
Diese muss jedoch keine zuverlässige Folge sein.

\begin{shaded}
\paragraph{Übung}
\label{par:ubung}

Es sollen codierende und nicht-codierende Abschnitte in DNA bestimmt werden.
Vorhanden sind:
\begin{itemize}
    \item einige codierende Sequenzen
    \item einige nicht-codierende Sequenzen
    \item große Menge an unbekannten Sequenzen
\end{itemize}
Gesucht ist ein Verfahren um codierende Regionen in DNA zu identifizieren,
gegeben eine DNA Sequenz.\\
Lösung:
\begin{itemize}
    \item Hidden-Markov-Model wie in BildX
    \item jeweils einen ML-Schäter für beide Zustandsmenge \{codierend,
        nicht-codierend\} (überwachtes Training)
    \item Übergangswahrscheinlichkeiten zwischen den Zustandsmengen \{codierend,
        nicht-codierend\} mittels unüberwachtem Training (Viterbi-Training oder
        Baum-Welch-Algorithmus)
    \item Backward-Algorithmus
        \begin{align*}
            P(Z_t \in \{a_c,c_c,g_c,t_c\} \mid  A_1=a_1,\ldots,A_n=a_n)=\\
            \sum\limits_{z\in a_c,c_c,g_c,t_c} P(Z_t=z\mid A_1=a_1,\ldots,A_n=a_n)
        \end{align*}
\end{itemize}
Wenn das Ergebnis größer als $\frac{1}{2}$ ist, ist es eine codierende Sequenz,
sonst uncodierend.
\end{shaded}

\begin{shaded}
\paragraph{Übung}
\label{par:ubung2}

Gegeben sei ein HMM.
In diesem werden die Zustände verdoppelt.
Wie muss sich die Länge der Trainigssequenz erhöhen, damit die gleiche Anzahl Zustandsübergänge vorhanden ist wie vorher.
Vereinfacht sein angenohmen das alle Zustandsübergänge die gleiche Wahrscheinlichkeiten besitzen.

Lösung:
\[\hat p(z,z') = \frac{n}{|z^2|}\]
Die Trainigssequenz muss 4-fach solang sein.
\end{shaded}

\subsection{Baum-Welch-Algorithmus}
Der Baum-Welch-Algorithmus wird benutzt, um die unbekannten Parameter eines Hidden Markov Models (HMM) zu finden.
Er nutzt dabei den Forward-Backward-Algorithmus zur Berechnung von Zwischenergebnissen, ist aber nicht mit diesem identisch.
Der Baum-Welch-Algorithmus ist ein erwartungsmaximierender Algorithmus.

Idee:
\begin{enumerate}
    \item HMM zufällig initialisieren
    \item Mit dem Backward-Algorithmus die Wahrscheinlickeit \[P(Z_{t} = j \mid  A_{1} = a_{1}, \ldots, A_{n} = a_{n})\]berechnen.
        Auf ähnliche Weise lassen sich \[P(Z_{t} = i, Z_{t+1} = j \mid A_{1} = a_{1}, \ldots, A_{n} = a_{n})\] berechnen.
        Damit lassen sich Erwartungswerte berechen für die Häufigkeit eines Zustandes, eines Zustandsübergangs oder einer Emission.
        Damit lassen sich Schätzer berechnen für die Parameter des HMM.
        Damit werden die Parameter des HMM geändert.
    \item Mehrfach iterieren, bis sich an den Parametern nichts mehr ändert, beste Lösung ausgeben (größte Wahrscheinlichkeit für Ausgabesequenz)
\end{enumerate}

\section{Lineare Regression}
\label{sec:lineare_regression}
Der Korrelationskoeffizient ist ein Maß für lineare Abhängigkeit zweier Zufallsvariablen.
Er kann Werte zwischen \(−1\) und \(+1\) annehmen.
Bei einem Wert von \(+1\) (bzw. \(−1\)) besteht ein vollständig positiver (bzw. negativer) linearer Zusammenhang zwischen den betrachteten Merkmalen.
Wenn der Korrelationskoeffizient den Wert 0 aufweist, hängen die beiden Merkmale überhaupt nicht linear voneinander ab.
\begin{align*}
	\varrho = \frac{Cov(x,y)}{\sqrt{Var X Var Y}}
\end{align*}

Wir wollen eine abhängige Größe \(Y\) als lineare Funktion in den Variablen \(f_1,\ldots, f_n\) (Features) darstellen
(\(w =\) Gewichtsvektor).
\begin{align*}
	y=w_0+\sum\limits_{i=1}^n w_i f_i
\end{align*}
Mit \(f_0=1\) gilt:
\begin{align*}
    y=\sum\limits_{i=0}^n w_i\cdot f_i= w \cdot f
\end{align*}
Für Trainingsdaten \(y^{(i)}, f^{(i)}\) bestimmen wir \(w\) so, dass der mittlere quadratische Fehler
\begin{align*}
    \sum\limits_i {(w\cdot f^{(i)}-y^{(i)})}^2
\end{align*}
minimiert wird.
Eine exakte Lösung ist möglich.


\section{Logistische Regression}
\label{sub:logistische_regression}
Für eine binäre Zielgröße \(y\) wollen wir \(P(y=true \mid f)\)\footnote{Vereinfachte Notation} bestimmen.
Da \(w\cdot f\) jedoch beliebige Werte annehmen kann, betrachten wir den Quotienten (Odds)
\begin{align*}
	\frac{P(y=true\mid f)}{1-P(y=true\mid f)}
\end{align*}
Dieser nimmt Werte $\geq 0$ an. Ansatz daher:
\begin{align}
\ln \frac{P(y=true\mid f)}{1-P(y=true\mid f)} = w\cdot f. \tag{1}\label{eq:one}
\end{align}
Damit folgt:
\begin{align*}
	\ln \frac{P(y=true\mid f)}{1-P(y=true\mid f)} &= w\cdot f\\
	\frac{P(y=true\mid f)}{1-P(y=true\mid f)} &= e^{w\cdot f}\\
	P(y=true\mid f) &=e^{w\cdot f}\cdot (1-P(y=true\mid f))\\
	P(y=true\mid f) &=e^{w\cdot f} - e^{w\cdot f}P(y=true\mid f)\\
	P(y=true\mid f) + e^{w\cdot f}P(y=true\mid f)&=e^{w\cdot f} \\
	P(y=true\mid f)\cdot (1+ e^{w\cdot f}) &=e^{w\cdot f} \\
	P(y=true\mid f) &=\frac{e^{w\cdot f}}{1+ e^{w\cdot f}} \\
	P(y=true\mid f) &=\frac{1}{1+ e^{-w\cdot f}} \tag{2}\label{eq:two}
\end{align*}
Die Funktion \(x\mapsto \frac{1}{1+e^{-x}}\) heißt logistische Funktion.
Zur Klassifikation einer Beobachtung $f$ verwenden wir den Ansatz
\begin{align*}
	P(y=true \mid f)> P(y=false \mid f)
\end{align*}
und damit
\begin{align*}
	\frac{P(y=true\mid f)}{1-P(y=true\mid f)} > 1
\end{align*}
Mit \cref{eq:one} erhalten wir daraus
\begin{align*}
    w\cdot f>0
\end{align*}
\paragraph{Geometrische Interpretation:}
\label{par:geometrische_interpretation_}

$w\cdot f=0$ ist die Gleichung einer Hyperebene.

Zum lernen der Gewichte verwenden wir einen ML-Ansatz
\begin{align*}
    \hat w & = \arg\max_w \prod\limits_i P(y=y^{(i)}\mid f^{(i)})\\
    \hat w & = \arg\max_w \sum\limits_i \log P(y=y^{(i)}\mid f^{(i)})\\
    \hat w & = \arg\max_w \left(\sum\limits_i y^{(i)}\log P(y=1\mid
        f^{(i)}) + \sum\limits_i (1-y^{(i)})\log P(y=0\mid f^{(i)})\right)\\
    \hat w & = \arg\max_w \left(\sum\limits_i y^{(i)} \log
    \left(\frac{1}{1+e^{-wf}}\right)+
    \sum\limits_i (1-y^{(i)}) \log \left(\frac{e^{-wf}}{1+e^{-wf}}\right)\right)
\end{align*}
Um diese Gleichung zu lösen werden numerische Verfahren (konvexe Optimierung) verwendet.


\section{Multinomiale logistische Regression}
\label{sec:multinomiale_logistische_regression}
Wir verallgemeinern die logistische Regression auf eine Menge von Klassen $C$.
Mit dem Spezialfall \(C=\{true,false\}\) haben wir ermittelt
\begin{equation*}
	\begin{aligned}
		P(y=true\mid x) &=\frac{1}{1+e^{-wf}}	&=\frac{e^{wf}}{1+e^{wf}} 	&=\frac{1}{z}e^{wf}\\
		P(y=false\mid x)&=\frac{1}{1+e^{wf}} 	&							&=\frac{1}{z}e^{0}\\
	\end{aligned}
\end{equation*}
Mit $w_{true} = w,\; w_{false}=0$ lässt sich dies schreiben als
\begin{align*}
    P(Y=true\mid x)=\frac{1}{z}e^{w_{\text{true}}f}\\
    P(Y=false\mid x)=\frac{1}{z}e^{w_{\text{false}}f}.
\end{align*}
Ferner ist
\begin{align*}
    z=e^{w_{true}f}+e^{f_{false}f}
\end{align*}
ein Normierungsfaktor.
Ansatz für $c\in C$ daher
\begin{align*}
    P(c\mid x)=\frac{1}{z}e^{w_cf} \qquad \text{mit } z=\sum\limits_{c\in
        C}e^{w_cf}
\end{align*}
Zur Klassifikation verwenden wir
\begin{align*}
    \hat c =\argmax_{c\in C}(P(c\mid x))
\end{align*}


\subsubsection{Softmax-Funktion}
\label{ssub:softmax_funktion}
Die Softmax-Funktion ist
\begin{align*}
    \softmax(c,C)= \frac{e^{x_c}}{\sum\limits_{c\in C}e^{x_c}}
\end{align*}
Es gilt: \label{par:es_gilt_}
\begin{itemize}
	\item $0 < \softmax < 1$
	\item Für $x_c \ll \max\{x_c\mid c\in C\}$ gilt $\softmax \approx 0$.
	\item Für $x_c \ll x_{c'},$ für $c' \neq c$ ist $\softmax \approx 1$.
\end{itemize}
Für Werte \(x_c\), die nicht zu dicht zusammen liegen gilt
\begin{align*}
\softmax(c,C)
\approx \begin{cases}1,& \text{ für }c=\argmax\{x_c\mid c\in C\}\\
0, & \text{ sonst }\end{cases}
\end{align*}
Die softmax Funktion ist differenzierbar.
Die Klassifikationswahrscheinlichkeiten lassen sich damit schreiben als
\begin{align*}
    P(c\mid x)=\softmax(c,\{w_c\cdot f \mid c\in C\})
    \tag{*}\label{eq:softmax}
\end{align*}


\begin{shaded}
\paragraph{Übung}
\label{par:ubungsm}
\begin{enumerate}
    \item Zeigen Sie, dass softmax eine Verallgemeinerung der logistischen
        Funktion ist.
    \begin{align*}
        \logistic(x)=\frac{1}{1+e^{-x}}=\frac{e^x}{1+e^x}=\frac{e^x}{e^x+e^0}=\frac{e^x}{\sum\limits_C
        e^{x_c}}
    \end{align*}
    \item Zeigen Sie, dass für jedes $K$ gilt: $\softmax(c,\{x_c\mid c\in
        C\})=\softmax(c,\{x_c+K\mid c\in C\})$.
        \begin{align*}
            \softmax(c,\{x_c\mid c\in C\}) &= \frac{e^{x_c+K}}{\sum\limits_{c\in
                    C}e^{x_c+K}} = \frac{e^{x_c}\cdot e^K}{\sum\limits_{c\in
                    C}e^{x_c}\cdot e^K}=\frac{e^{x_c}}{\sum\limits_{c\in
                    C}e^{x_c}}\\ 
            &= \softmax(c,\{x_c+K\mid c\in C\}) \tag{u2}\label{eq:u2}
        \end{align*}
\end{enumerate}
\end{shaded}
Wegen \cref{eq:u2} können wir in \cref{eq:softmax} daher ein Gewicht auf 0 setzen, durch \(w_{c'} = w_c-w_{c^*}\) für ein beliebiges $c^*\in C$.
Damit sind nur noch \(|C|-1\) Gewichte zu lernen.
Damit
\begin{align*}
    P(c\mid x)=\frac{e^{w_{c'}}f}{1+\sum\limits_{c\neq c^*}e^{x_c}f}.
\end{align*}

\begin{shaded}
\paragraph{Beispiel}
\label{par:beispielsw}
Erkennen von Objekten

Es sollen Bilder von Nägeln, Schrauben und Muttern korrekt klassifiziert werden.

Features:
\begin{align*}
    f_1(x)&=\begin{cases}1, & \text{ wenn in der Bildmitte ein helles Pixel }\\
        0, & \text{sonst} \end{cases}\\
    f_2(x)&=\frac{\text{Bildbreite}}{\text{Bildhöhe}}\\
    f_3(x)&=\text{Anzahl an Kanten}
\end{align*}
3 Gewichte für die 3 Features lernen und mit multinomialer Regression
klassifizieren.
\end{shaded}

\paragraph{Interpretation der logistischen Regression} als Maximum-Entropy-Model
\label{par:interpretation_der_logistischen_regression}

Bereits bekant: Entropie ist ein Maß für den Informationsgehalt, aber auch der
Unsicherheit.

Max-Entropie-Prinzip: Unter mehreren möglichen Verteilungen wählen wir jene mit
maximaler Entropie.

\begin{shaded}
\paragraph{Beispiel}
\label{par:beispielme}
Wir wissen, dass eine Zufallsvariable $X$ die Werte 1,2,3,4 annimmt.
Ein MaxEnt-Modell dazu ist die Gleichverteilung auf \{1,2,3,4\}.
Wenn nun zusätzlich bekannt ist, dass $P(X=1)=\frac{1}{2}$, erhalten wir als MaxEnt-Modell
\begin{tabular}[t]{cccc}
	1 & 2 & 3 & 4 \\\hline
	$\frac{1}{2}$ & $\frac{1}{6}$ & $\frac{1}{6}$ & $\frac{1}{6}$
\end{tabular}
Sei ferner $P(X=3 \lor X=4)=\frac{1}{4}$ bekannt:
\begin{tabular}[t]{cccc}
	1 & 2 & 3 & 4 \\\hline
		$\frac{1}{2}$ & $\frac{1}{4}$ & $\frac{1}{8}$ & $\frac{1}{8}$
\end{tabular}
\end{shaded}

Man kann zeigen, dass eine multinomiale logistische Regression eine Lösung des Optimierungsproblems
\begin{align*}
\argmax_{p\in M} H(p)
\end{align*}
liefert.
Deshalb wird die logistische Regression auch als MaxEnt bezeichnet.


\section{Naive Bayes Klassifikator}
\label{sec:naive_bayes_klassifikator}
Ansatz:
\begin{align*}
    P(c,x)=\frac{P(c)P(x\mid c)}{P(x)}
\end{align*}
Klassifikationsregel:
\begin{align*}
    \hat c=\argmax\limits_{c\in C} P(c\mid x) \tag{*}\label{eq:klassbayes}
\end{align*}
Dabei ist $x=(x_1,x_2,\dots,x_n)$.
Wenn die Zufallsvariablen $x_i$ bedingt unabhängig gegben \(c\) sind, gilt
\begin{align*}
    P(x\mid c)=P(x_1,\dots,x_n\mid c)=\prod_i P(x_i\mid c)
\end{align*}
so dass sich \cref{eq:klassbayes} vereinfacht zu
\begin{align*}
    \hat c= \argmax_{c\in C}p(c)\prod_i P(x_i\mid c)
\end{align*}
Die verschiedenen Varianten des Naive bayes unterscheiden sich in der Modellierung von $P(x_i\mid c)$ (normalverteilt, multinomialverteilt, Bernoulli, etc).


\end{document}


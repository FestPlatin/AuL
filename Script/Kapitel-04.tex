\chapter{Graphalgorithmen}
Aus der Vorlesung Künstliche Intelligenz sind bereits die Algorithmen Breiten- und Tiefensuche bekannt.
\begin{lstlisting}[language=java, caption={Beispiel Algorithmus für die Breitensuche}]
boolean bfs(start, goal) {

	// Anfangs sind keine Knoten besucht
	for(v in V)
		discovered[v] = false;
	
	// Mit Start-Knoten beginnen
	queue.enqueue(start)
	discovered[start] = true;
	
	while(!queue.isEmpty()){
		// Erstes Element von der queue nehmen
		u = queue.dequeue;
		
		// Testen ob Zielknoten gefunden
		if(u == goal)
			return true;

		// alle Nachfolge-Knoten, ...
		for(v in adj[u]))
			// ... die noch nicht besucht wurden ...
			if(!discovered[v]){
				// ... zur queue hinzufuegen ...
				queue.enqueue(v);
				// ... und als bereits gesehen markieren
				discovered[u] = true;
			}
	}
	return false;
}
\end{lstlisting}
\newpage
Für dieses Beispiel fallen folgende Laufzeiten an:
\begin{itemize}
	\item Zeile 4-5: \(\mathcal{O}(|V|)\)
	\item Zeile 8-9: \(\mathcal{O}(1)\)
	\item Zeile 13-17: \(\mathcal{O}(1)\)
	\item Zeile 20-27: \(\mathcal{O}(\deg(u))\)
	\item Zeile 29: \(\mathcal{O}(1)\)
\end{itemize}
\(\deg(u)\) steht dabei für den Grad des Knotens (=Anzahl der Nachbarn).
Ein Knoten \(v\) hat den Grad \(k\) wenn \(v\) mit genau \(k\) anderen Knoten verbunden ist.
Wir schreiben dafür \(\deg(u) = k\).

\begin{shaded}
	\noindent
	\textbf{Satz:} Die Laufzeit der Breitensuche liegt in \(\mathcal{O}(|V|+|E|)\).

	Dabei ist \(|V|\) die Anzahl der Knoten (Vertex) und \(|E|\) die Anzahl der Kanten (Edge) im Graphen.
\end{shaded}

\paragraph{Beweis:}
Das initialisieren des Feldes discovered benötigt die Zeit \(\mathcal{O}(|V|)\).
Um die unbesuchten Nachbarn des Knotens \(u\) zu bestimmen fällt der Aufwand \(\mathcal{O}(\deg(u))\) an.
Da jeder Knoten höchstens einmal aus der Warteschlange entnommen wird, wird auch die while Schleife für jeden Knoten höchstens einmal durchlaufen.
Der gesamte Aufwand ist damit
\[\mathcal{O}(|V|) + \sum\limits_{u \in V} \mathcal{O}(\deg(u) = \mathcal{O}(|V|) + \mathcal{O}(|E|) = \mathcal{O}(|V| + |E|)\]
Die Tiefensuche lässt sich implentieren wie die Breitensuche, wenn anstelle der Warteschlange ein Stack verwendet wird.
Die Laufzeit ist gleich der Breitensuche: \(\mathcal{O}(|V| + |E|)\).


\section{Verallgemeinerung der A*-Suche}
Hierbei wird eine heuristische Bewertungsfunktion \(f\) verwendet um Knoten einzusortieren.
Die heuristische Bewertungsfunktion hat die Gestalt
\[f(v) = g(v) + h(v)\]
wobei
\begin{itemize}
	\item \(g(v)\) die Kosten bis zum Knoten v
	\item \(h(v)\) eine zulässige Kostenschätzfunktion
\end{itemize}
sind.
Eine Kostenschätzfunktion \(h\) ist zulässig, wenn sie die Kosten zum Ziel nicht überschätzt.
\begin{figure}[htbp]
	\centering
	\begin{tikzpicture}[node distance=2cm]
		\draw (0,0) circle[radius=4pt] node[align=center, below, yshift=-4mm] {Start};
		\draw (2,0) circle[radius=4pt] node[align=center, below, yshift=-4mm] {v};
		\draw [zigzag] (0.2,0) -- (1.8,0);
		\draw (4,0) circle[radius=4pt] node[align=center, below, yshift=-4mm] {Ziel};
		\draw [dotted] (2.2,0) -- (3.8,0);
		\draw[thick,decorate,decoration={brace,amplitude=12pt}] (2,-.8) -- (0,-.8) node[midway,below, yshift=-12pt]{\(g(v)\)};
		\draw[thick,decorate,decoration={brace,amplitude=12pt}] (4,-.8) -- (2,-.8) node[midway,below, yshift=-12pt]{\(h(v)\)};
	\end{tikzpicture}
	\caption{Veranschaulichung der heuristische Bewertungsfunktion}
\end{figure}
\newpage
Bei einer Navigation könnten die Funktionen wie folgt definiert sein:
\begin{itemize}
	\item \(g(v)\): Strecke von Start bis v
	\item \(h(v)\): Luftlinienentfernung von v zum Ziel
\end{itemize}

\subsubsection{Übung}
Für ein Streckennetz sind Entfernungen und durchschnittliche Geschwindigkeiten bekannt.
Wie kann die schnellste Route von A nach B gefunden werden?
\begin{itemize}
	\item \(g(v)\): Zeit für Strecke von Start bis v (\(t=\frac{s}{v}\))
	\item \(h(v)\): \(t=\frac{\textrm{Luftlinie bis zum Ziel}}{\textrm{maximale Geschwindigkeit der verbleibenden Kanten}}\)
\end{itemize}


\section{Topologisches Sortieren}
\begin{shaded}
  \noindent
  \textbf{Def.:} Ein DAG (Directed acyclic graph) ist ein gerichteter Graph, der keine gerichteten Kreise enthält.
		Eine topologische Sortierung eines DAG \(G = (V,E)\) ist eine Abbildung
		\[f: V \rightarrow \mathbb{N} \textrm{ mit } f(u) < f(v) \textrm{ für } (u,v) \in E\]
\end{shaded}
\begin{figure}[htbp]
	\begin{center}
	 	\begin{tikzpicture}[sibling distance=5mm]
			\node[state] (1) at (1,0) {1};
			\node[state] (2) at (0,-1) {2};
			\node[state] (3) at (2,-1) {4};
			\node[state] (4) at (-1,-2) {4};
			\draw[->] (1) -- (2);
			\draw[->] (1) -- (3);
			\draw[->] (2) -- (3);
			\draw[->] (2) -- (4);
		\end{tikzpicture}
		\caption{Beispiel für ein Directed acyclic graph (DAG)}
	\end{center}
\end{figure}

Jeder vollständige Graph oder Kreis lässt sich nicht topologisch sortieren.
Eine topologische Sortierung kann durch eine Tiefensuche bestimmt werden.
\newpage
\begin{lstlisting}[language=java,caption={Pseudocode für eine topologische Sortierung}]
topsort:
	for (v in V)
		markiere v mit weiss
	for (v in V)
		tiefensuche(v)

tiefensuche(v):
	v grau:
		Fehler("Kreis vorhanden")
	v weiss:
		markiere v mit grau
		for (u in adj[v])
			tiefensuche(u)
		markiere v mit schwarz
		fuege v an den Kopf einer Liste
\end{lstlisting}

\begin{shaded}
	\noindent
	\textbf{Satz:} Für jeden DAG \(G=(V,E)\) erzeugt Topsort eine topologische Sortierung von \(G\)
\end{shaded}

\paragraph{Beweis:}
Sei \((u,v) \in E\).
In u und in v werden je eine Tiefensuche gestartet.
Die in v gestartete Tiefensuche endet früher als die in u gestartete Tiefensuche.
Daher wird u links von v in die Liste eingefügt und erhält daher eine kleinere Nummer als v.
\chapter{O-Notation, Laufzeitanalyse}
Man kann den Zeitaufwand von Algorithmen nicht eindeutig bestimmen.
Viel zu viele Faktoren (Hardware, parallel laufende Programme, Eingabereihenfolge, \dots) spielen eine Rolle, so dass man mit normalen Mitteln niemals eine genaue und allgemeine Aussage über die benötigte Zeit machen kann.
Es werden nun nicht mehr die benötigten Zeiten, sondern die benötigten ``greifbaren'' Schritte bei einer bestimmten Eingabelänge n beschrieben.
Somit können Programme in Klassen (konstant, logarithmisch, lineas, polynomial, exponentiell, u.a.) eingeteilt werden.

\section{O-Notation}
Wir verwenden die \(\mathcal{O}\)-Notation um die Laufzeit und den Platzbedarf von Algorithmen anzugeben.
Dazu betrachten wir die maximale Anzahl Schritte, die ein Algorithmus ausführt.

\subsubsection{Beispiel}
Es soll die Laufzeit der lineare Suche berechnet werden.
\begin{lstlisting}[language=java, caption={Pseudocode zur Berechnung der Laufzeit}]
for (k := 1 to n) {
	if (a[k] == gesuchter Wert)
		return true;
}
return false;
\end{lstlisting}
Lösung:
\begin{eqnarray*}
	LZ	&\leq& n \cdot c_{1} + c_{2} \\
		&\leq& n \cdot c + c  \textrm{ wobei } c= \max\{c_{1},c_{2}\} \\
		&\leq& n \cdot c + n \cdot c	\\
		&=& 2c \cdot n \in \mathcal{O}(n)
\end{eqnarray*}
Die Laufzeit der linearen Suche ist \(\leq c \cdot n\), wobei \(c\) eine Konstante ist, die von der Implentierung und dem Computer abhängt.

\begin{shaded}
	\noindent
	\textbf{Def.:} Für eine Funktion \(f \geq 0\) ist \(\mathcal{O}(f)\) die Menge aller Funktionen \(g\) mit 
		\[0 \leq g(n) \leq c \cdot f(n)\] 
		für eine Konstante \(c > 0\) für alle hinreichend großen \(n\).
		
		\(\mathcal{O}(f) = \{ g\:|\: 0 \leq g(n) \leq c \cdot f(n)\) für ein \(c > 0\) und allen großen \(n \in \mathcal{N}\}\)
\end{shaded}
Die \(\mathcal{O}\)-Notation stellt somit die maximale Laufzeit (Worst Case Laufzeit) eines Algorithmuses dar.
Die Funktion \(g(n)\) ist die konkrete Laufzeit einer gegebenen Implentierung.
\begin{figure}[htbp]
	\begin{center}
		\begin{tikzpicture}[
				%We set the scale and define some styles
				scale=1.5,
				axis/.style={very thick, ->, >=stealth'},
				important line/.style={thick},
				dashed line/.style={dashed, thick},
				every node/.style={color=black,}
			]
			
			% Important coordinates are defined
			\coordinate (x11) at (0,0);
			\coordinate (x12) at (2,1.5);
			\coordinate (x21) at (0,0.2);
			\coordinate (x22) at (2.5,1.3);
		
			\begin{scope}
			\shade[top color=white, bottom color=red]
			(x12) parabola bend (0,0) (2,0);
			\end{scope}

			% axis
			\draw[axis] (0,0)  -- (3,0) node(xline)[right] {n};
			\draw[axis] (0,0) -- (0,2) node(yline)[above] {};
			
			% J curve is drawn
			\draw[important line]
			(x11) parabola (x12) node[above] {$c \cdot f$}
			(x21) parabola (x22) node[right] {g};

			\draw[dashed line] (1.8,.3) -- (2.5,.5) node[right] {$\mathcal{O}(f)$};
			\draw[thick] (1,.38) circle (1.5pt) node[above] {$n_{0}$};
		\end{tikzpicture}
	\end{center}
	\label{img:ONotation}
	\caption{Grafische Darstellung der \(\mathcal{O}\)-Funktion}
\end{figure}

Es gelten folgende Rechenregeln:
\begin{itemize}
	\item \(\mathcal{O}(f) + \mathcal{O}(g) = \mathcal{O}(\max\{f,g\})\)
	\item \(\mathcal{O}(f) + \mathcal{O}(g) = \mathcal{O}(f + g)\)
	\item \(\mathcal{O}(f) \cdot \mathcal{O}(g) = \mathcal{O}(f \cdot g)\)
	\item \(\mathcal{O}(c \cdot f) = \mathcal{O}(f)\) (für alle \(c \geq 0)\)
	\item \(f \leq g \Rightarrow \mathcal{O}(f) \subseteq \mathcal{O}(g)\)
	\item \(\mathcal{O}(c) = \mathcal{O}(1)\)
	\item \(c = \mathcal{O}(1)\)
\end{itemize}


\subsubsection{Übung}
\begin{eqnarray*}
	0	&\leq& 17n^3 + 5n^2 + 2n + 8 \\
		&\leq& 17n^3 + 5n^3 + 2n^3 + 8n^3 \\
		&\leq& 32n^3 \in \mathcal{O}(n^3)
\end{eqnarray*}

\subsubsection{Übung}
Berechnen Sie die Laufzeit des folgenden Codes:
\begin{lstlisting}[language=java, caption={Pseudocode zur Berechnung der Laufzeit}]
	for (k := 1 to n-1)
		for (l := k+1 to n)
			if (a[k] == a[l])
				return Duplikat vorhanden;
	return Kein Duplikat vorhanden;
\end{lstlisting}
\begin{enumerate}
	\item Möglichkeit zur Abschätzung der Laufzeit
		\begin{eqnarray*}
		LZ 	&\leq& n \cdot n \cdot c + c'	\\
			&\leq& n^2 \cdot c + n^2 \cdot c'	\\
			&\leq& (c+c') n^2				\\
			&\leq& \mathcal{O}(n^{2})
		\end{eqnarray*}
	\item Möglichkeit zur genaueren Abschätzung der Laufzeit
	\begin{eqnarray*}
		LZ	&\leq& \binom{n}{2} \cdot c + c' \\
			&\leq& \binom{n}{2} \cdot c + \binom{n}{2} \\
			&=& \binom{n}{2}(c+1)	\\
			&=& \frac{n(n-1)}{2}(c+1)	\\
	\curvearrowright LZ &\leq& n^{2} \cdot \frac{c+1}{2} \in \mathcal{O}(n^{2})
	\end{eqnarray*}
\end{enumerate}

\newpage
\section{Omega-Notation}
\label{sec:OmegaNotation}
Die Omega-Notation beschreibt die untere Schranke, d.h. wie lange ein Algorithmus bzw. ein Programm mindestens läuft (Best Case Laufzeit).
\begin{shaded}
	\noindent
	\textbf{Def.:} \(\Omega(f) = \{ g \:|\) Es gibt ein \(c > 0\), sodass \(0 \leq c \cdot f(n) \leq g(n)\) für alle großen \(n\) gilt.\(\}\)
\end{shaded}

\begin{figure}[htbp]
	\begin{center}
		\begin{tikzpicture}[
				%We set the scale and define some styles
				scale=1.5,
				axis/.style={very thick, ->, >=stealth'},
				important line/.style={thick},
				dashed line/.style={dashed, thick},
				every node/.style={color=black,}
			]
			
			% Important coordinates are defined
			\coordinate (x11) at (0,0);
			\coordinate (x12) at (2,1.5);
			
			\begin{scope}
			\shade[top color=white, bottom color=red]
			(x12) parabola bend (0,0) (0,1.5);
			\end{scope}
			
			% axis
			\draw[axis] (0,0)  -- (3,0) node(xline)[right] {n};
			\draw[axis] (0,0) -- (0,2) node(yline)[above] {};
			
			% J curve is drawn
			\draw[important line]
			(x11) parabola (x12) node[above] {$c \cdot f$};

			\draw[dashed line] (1,1.3) -- (1,1.7) node[above] {$\Omega(f)$};
		\end{tikzpicture}
	\end{center}
	\label{img:OmegaNotation}
	\caption{Grafische Darstellung der Omega-Funktion}
\end{figure}

\section{Theta-Notation}
\label{sec:ThetaNotation}
Die Theta-Notation dient dazu, gleichzeitg eine obere und eine untere Schranke zu definieren.
\begin{shaded}
	\noindent
	\textbf{Def.:} \( \Theta(f) = \mathcal{O} \cap \Omega(f)\)
\end{shaded}
\begin{figure}[htbp]
	\begin{center}
		\begin{tikzpicture}[
				%We set the scale and define some styles
				scale=1.5,
				axis/.style={very thick, ->, >=stealth'},
				important line/.style={thick},
				dashed line/.style={dashed, thick},
				every node/.style={color=black,}
			]
			
			% Important coordinates are defined
			\coordinate (x11) at (0,0);
			\coordinate (x12) at (1.5,1.5);
			\coordinate (x22) at (2.5,1.5);
			
			\begin{scope}
				\shade[top color=white, bottom color=red]
					(x22) parabola bend (0,0) (x12);
			\end{scope}
			
			% axis
			\draw[axis] (0,0)  -- (3,0) node(xline)[right] {n};
			\draw[axis] (0,0) -- (0,2) node(yline)[above] {};
			
			% J curve is drawn
			\draw[important line]
			(x11) parabola (x12) node[above] {$c_{1} \cdot f$};
			\draw[important line]
			(x11) parabola (x22) node[above] {$c_{2} \cdot f$};

			\draw[dashed line] (1.5,.8) -- (2.3,.8) node[right] {$\Theta(f))$};
		\end{tikzpicture}
	\end{center}
	\label{img:Theta}
	\caption{Grafische Darstellung der Theta-Funktion}
\end{figure}

\documentclass[a4paper,11pt]{exam}
%\printanswers % If you want to print answers
\noprintanswers % If you don't want to print answers
\addpoints % if you want to count the points
%\noaddpoints % if you don't want to count the points

\usepackage{color}					% defines a new color
\usepackage[utf8]{inputenc}			%UTF-8 Unterstützung (Kodierung)
\usepackage[ngerman]{babel}			%Deutsche Sprachdatei für Umlaute
\usepackage{amssymb}   				% Mathematische Symbole und Zeichen
\usepackage{amsmath}   				% Mathematische Symbole und Zeichen
\usepackage{listings}
\usepackage{tabularx}
\usepackage{tikz}
\usepackage{color}
\usepackage{array,paralist,ragged2e}		% Itemsize in Tabelle
\newcolumntype{i}[1]{%
  >{\vspace*{-.5\baselineskip}%
    \RaggedRight%
    \setdefaultleftmargin{1em}{}{}{}{}{}%
    \begin{compactitem}}
    p{#1}%
  <{\end{compactitem}%
    \vspace*{-\baselineskip}}
}

%\qformat{\textbf{Aufgabe \thequestion}\hfill\quad(\thepoints)}
\renewcommand{\solutiontitle}{\noindent\textbf{Lösung:}\enspace}
\pointsinrightmargin
\pointpoints{Punkt}{Punkte}
\cfoot[]{Seite \thepage}

\shadedsolutions % defines the style of the solution environment
\title{Algorithmen und Lernverfahren}

\author{Kay Förster}
\date{\today}

\printanswers

\begin{document}
\maketitle

\begin{questions}
	\question Berechnen Sie die Laufzeit des folgenden Algorithmuses:
		\[\sum_{m=0}^{k} a_m n^m \;\;\textrm{mit } a_k > 0, a_m \geq 0\]
	\begin{solution}
		\begin{eqnarray*}
			LZ	&\leq& \sum_{m=0}^{k} a_m n^m \\
				&\leq& a_0 n^0 + \dots + a_k n^k \\
				&\leq& a_k n^k \in \mathcal{O}(n^k)
		\end{eqnarray*}
	\end{solution}


	\question Zeigen Sie:
		\begin{parts}
			\part \(\log(n!) \in \mathcal{O}(n \log n)\)
			\part \(\binom{n}{k} \in \mathcal{O}(n^k)\)
			\part \(2^{n+\mathcal{O}(1)} \in \mathcal{O}(2^n)\)
		\end{parts}
	\begin{solution}
		\begin{parts}
			\part
				\begin{eqnarray*}
					\mathcal{O}(\log(n!)) &\subseteq& \mathcal{O}(n \log n) \\
					\mathcal{O}(\log(n \cdot n-1 \cdot \ldots \cdot n-(n-1))) &\subseteq&	\\
					\mathcal{O}(\log(n^n)) &\subseteq&	\\
				\end{eqnarray*}
			\part
			\part \[2^{n+\mathcal{O}(1)} \subseteq 2^{n+c} = 2^n \cdot 2^c \in \mathcal{O}(2^n)\]
		\end{parts}
	\end{solution}


	\question Zeigen Sie das \(\frac{n(n+1)}{2} \in \Theta(n^{2})\) gilt.
	\begin{solution}
		\begin{eqnarray*}
			\mathcal{O}(\frac{n(n+1)}{2}) = \mathcal{O}(n(n+1)) = \mathcal{O}(n) \cdot \mathcal{O}(n+1) = \mathcal{O}(n) \cdot \mathcal{O}(n) = \mathcal{O}(n^{2}) \\
			\frac{n(n+1)}{2} \geq \frac{n^{2}}{2} \in \Theta(n^{2}) \\
			\curvearrowright \mathcal{O}(n^{2}) \cap \Omega(n^{2}) = \Theta(n^{2})
		\end{eqnarray*}
	\end{solution}

	\question Zeigen Sie ob \(n \in \Theta(2^n)\) gilt.
	\begin{solution}
		\[n \in \Theta(2^n) = \mathcal{O}(2^n) \cap \underbrace{\Omega(2^n)}_{\textrm{Verletzt}}\]
		\(2^n\) wächst schneller als \(n\), somit kann die untere Schranke nicht eingehalten werden.
	\end{solution}
	
	\question Berechnen Sie die Entropie einer 
	\begin{parts}
		\part Gleichverteilung
		\part Verteilung mit \(p_1 = 1, p_k = 0 (k \neq 1)\)
	\end{parts}
	
	\begin{solution}
		\begin{parts}
			\part \(-[\frac{1}{n}\log\frac{1}{n} + \ldots + \frac{1}{n}\log\frac{1}{n}] = \log(\frac{1}{n})\)
			\part Die Entropie beträgt 0, da immer nur \(p_1\) ausgeben wird und nie ein anderes Zeichen.
		\end{parts}
	\end{solution}


	
	\question Gegeben sei die folgende Verteilung. Erstellen Sie eine Huffman-Codierung für das Wort ``Betriebssysteme''
	\begin{table}[htbp]
		\centering
		\begin{tabular}{l|cccccccc}
			\textbf{Zeichen}			& b	& e & i& m & r & s & t & y\\
			\textbf{Wahrscheinlichkeit}	& 0,02 & 0,17 & 0,08 & 0,03 & 0,07 & 0,07 & 0,06 & 0,00
		\end{tabular}
	\end{table}



\end{questions}
\end{document}